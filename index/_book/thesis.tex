% This is the Reed College LaTeX thesis template. Most of the work
% for the document class was done by Sam Noble (SN), as well as this
% template. Later comments etc. by Ben Salzberg (BTS). Additional
% restructuring and APA support by Jess Youngberg (JY).
% Your comments and suggestions are more than welcome; please email
% them to cus@reed.edu
%
% See http://web.reed.edu/cis/help/latex.html for help. There are a
% great bunch of help pages there, with notes on
% getting started, bibtex, etc. Go there and read it if you're not
% already familiar with LaTeX.
%
% Any line that starts with a percent symbol is a comment.
% They won't show up in the document, and are useful for notes
% to yourself and explaining commands.
% Commenting also removes a line from the document;
% very handy for troubleshooting problems. -BTS

% As far as I know, this follows the requirements laid out in
% the 2002-2003 Senior Handbook. Ask a librarian to check the
% document before binding. -SN

%%
%% Preamble
%%
% \documentclass{<something>} must begin each LaTeX document
\documentclass[12pt,twoside]{reedthesis}
% Packages are extensions to the basic LaTeX functions. Whatever you
% want to typeset, there is probably a package out there for it.
% Chemistry (chemtex), screenplays, you name it.
% Check out CTAN to see: http://www.ctan.org/
%%
\usepackage{graphicx,latexsym}
\usepackage{amsmath}
\usepackage{amssymb,amsthm}
\usepackage{longtable,booktabs,setspace}
\usepackage{chemarr} %% Useful for one reaction arrow, useless if you're not a chem major
\usepackage[hyphens]{url}
% Added by CII
\usepackage{hyperref}
\usepackage{lmodern}
\usepackage{float}
\floatplacement{figure}{H}
% End of CII addition
\usepackage{rotating}

% Next line commented out by CII
%%% \usepackage{natbib}
% Comment out the natbib line above and uncomment the following two lines to use the new
% biblatex-chicago style, for Chicago A. Also make some changes at the end where the
% bibliography is included.
%\usepackage{biblatex-chicago}
%\bibliography{thesis}


% Added by CII (Thanks, Hadley!)
% Use ref for internal links
\renewcommand{\hyperref}[2][???]{\autoref{#1}}
\def\chapterautorefname{Chapter}
\def\sectionautorefname{Section}
\def\subsectionautorefname{Subsection}
% End of CII addition

% Added by CII
\usepackage{caption}
\captionsetup{width=5in}
% End of CII addition

% \usepackage{times} % other fonts are available like times, bookman, charter, palatino

% Syntax highlighting #22
  \usepackage{color}
  \usepackage{fancyvrb}
  \newcommand{\VerbBar}{|}
  \newcommand{\VERB}{\Verb[commandchars=\\\{\}]}
  \DefineVerbatimEnvironment{Highlighting}{Verbatim}{commandchars=\\\{\}}
  % Add ',fontsize=\small' for more characters per line
  \usepackage{framed}
  \definecolor{shadecolor}{RGB}{248,248,248}
  \newenvironment{Shaded}{\begin{snugshade}}{\end{snugshade}}
  \newcommand{\KeywordTok}[1]{\textcolor[rgb]{0.13,0.29,0.53}{\textbf{#1}}}
  \newcommand{\DataTypeTok}[1]{\textcolor[rgb]{0.13,0.29,0.53}{#1}}
  \newcommand{\DecValTok}[1]{\textcolor[rgb]{0.00,0.00,0.81}{#1}}
  \newcommand{\BaseNTok}[1]{\textcolor[rgb]{0.00,0.00,0.81}{#1}}
  \newcommand{\FloatTok}[1]{\textcolor[rgb]{0.00,0.00,0.81}{#1}}
  \newcommand{\ConstantTok}[1]{\textcolor[rgb]{0.00,0.00,0.00}{#1}}
  \newcommand{\CharTok}[1]{\textcolor[rgb]{0.31,0.60,0.02}{#1}}
  \newcommand{\SpecialCharTok}[1]{\textcolor[rgb]{0.00,0.00,0.00}{#1}}
  \newcommand{\StringTok}[1]{\textcolor[rgb]{0.31,0.60,0.02}{#1}}
  \newcommand{\VerbatimStringTok}[1]{\textcolor[rgb]{0.31,0.60,0.02}{#1}}
  \newcommand{\SpecialStringTok}[1]{\textcolor[rgb]{0.31,0.60,0.02}{#1}}
  \newcommand{\ImportTok}[1]{#1}
  \newcommand{\CommentTok}[1]{\textcolor[rgb]{0.56,0.35,0.01}{\textit{#1}}}
  \newcommand{\DocumentationTok}[1]{\textcolor[rgb]{0.56,0.35,0.01}{\textbf{\textit{#1}}}}
  \newcommand{\AnnotationTok}[1]{\textcolor[rgb]{0.56,0.35,0.01}{\textbf{\textit{#1}}}}
  \newcommand{\CommentVarTok}[1]{\textcolor[rgb]{0.56,0.35,0.01}{\textbf{\textit{#1}}}}
  \newcommand{\OtherTok}[1]{\textcolor[rgb]{0.56,0.35,0.01}{#1}}
  \newcommand{\FunctionTok}[1]{\textcolor[rgb]{0.00,0.00,0.00}{#1}}
  \newcommand{\VariableTok}[1]{\textcolor[rgb]{0.00,0.00,0.00}{#1}}
  \newcommand{\ControlFlowTok}[1]{\textcolor[rgb]{0.13,0.29,0.53}{\textbf{#1}}}
  \newcommand{\OperatorTok}[1]{\textcolor[rgb]{0.81,0.36,0.00}{\textbf{#1}}}
  \newcommand{\BuiltInTok}[1]{#1}
  \newcommand{\ExtensionTok}[1]{#1}
  \newcommand{\PreprocessorTok}[1]{\textcolor[rgb]{0.56,0.35,0.01}{\textit{#1}}}
  \newcommand{\AttributeTok}[1]{\textcolor[rgb]{0.77,0.63,0.00}{#1}}
  \newcommand{\RegionMarkerTok}[1]{#1}
  \newcommand{\InformationTok}[1]{\textcolor[rgb]{0.56,0.35,0.01}{\textbf{\textit{#1}}}}
  \newcommand{\WarningTok}[1]{\textcolor[rgb]{0.56,0.35,0.01}{\textbf{\textit{#1}}}}
  \newcommand{\AlertTok}[1]{\textcolor[rgb]{0.94,0.16,0.16}{#1}}
  \newcommand{\ErrorTok}[1]{\textcolor[rgb]{0.64,0.00,0.00}{\textbf{#1}}}
  \newcommand{\NormalTok}[1]{#1}

% To pass between YAML and LaTeX the dollar signs are added by CII
\title{Tools for Understanding Taxicab and E-Hail Service Use in New York City}
\author{Wencong (Priscilla) Li}
% The month and year that you submit your FINAL draft TO THE LIBRARY (May or December)
\date{May 2018}
\division{}
\advisor{Benjamin Baumer}
\institution{Smith College}
\degree{Bachelor of Arts}
%If you have two advisors for some reason, you can use the following
% Uncommented out by CII
% End of CII addition

%%% Remember to use the correct department!
\department{Statistical and Data Sciences}
% if you're writing a thesis in an interdisciplinary major,
% uncomment the line below and change the text as appropriate.
% check the Senior Handbook if unsure.
%\thedivisionof{The Established Interdisciplinary Committee for}
% if you want the approval page to say "Approved for the Committee",
% uncomment the next line
%\approvedforthe{Committee}

% Added by CII
%%% Copied from knitr
%% maxwidth is the original width if it's less than linewidth
%% otherwise use linewidth (to make sure the graphics do not exceed the margin)
\makeatletter
\def\maxwidth{ %
  \ifdim\Gin@nat@width>\linewidth
    \linewidth
  \else
    \Gin@nat@width
  \fi
}
\makeatother

\renewcommand{\contentsname}{Table of Contents}
% End of CII addition

\setlength{\parskip}{0pt}

% Added by CII
  %\setlength{\parskip}{\baselineskip}
  \usepackage[parfill]{parskip}

\providecommand{\tightlist}{%
  \setlength{\itemsep}{0pt}\setlength{\parskip}{0pt}}

\Acknowledgements{
I would love to thank my thesis advisor Benjamin Baumer, Assistant
Professor of Statistical and Data Sciences at Smith College, for
encouraging me to challenge myself and guiding me through this project.
I want to thank Jordan Crouser for being my second reader and help me to
revise my paper. I also want to thank all my friends and my roommates
for their love and support.
}

\Dedication{

}

\Preface{

}

\Abstract{
Yellow Taxi Cabs are widely recognized as the icons of New York City.
The New York City Taxi \& Limousine Commission (TLC) provides publicly
accessible yellow and green taxi trip records (N. T. staff, 2009a). Each
taxi trip record is like a little piece of a gigantic puzzle, and all
together they tell a story of what happens in New York City. This thesis
presents a more efficient and easy-to-use way for users to retrieve trip
records information of both New York City taxi and other ride-sharing
services, such as Uber and Lyft, in New York City. By analyzing trip
records of New York City's iconic yellow taxi, we seek answers to
questions that are commonly asked by taxi drivers, passengers, and TLC
officials to help all three parties to improve their services or
experiences.
}

	\usepackage{setspace}\doublespacing
% End of CII addition
%%
%% End Preamble
%%
%

\usepackage{amsthm}
\newtheorem{theorem}{Theorem}[chapter]
\newtheorem{lemma}{Lemma}[chapter]
\newtheorem{corollary}{Corollary}[chapter]
\newtheorem{proposition}{Proposition}[chapter]
\theoremstyle{definition}
\newtheorem{definition}{Definition}[chapter]
\theoremstyle{definition}
\newtheorem{example}{Example}[chapter]
\theoremstyle{definition}
\newtheorem{exercise}{Exercise}[chapter]
\theoremstyle{remark}
\newtheorem*{remark}{Remark}
\newtheorem*{solution}{Solution}
\begin{document}

% Everything below added by CII
  \maketitle

\frontmatter % this stuff will be roman-numbered
\pagestyle{empty} % this removes page numbers from the frontmatter
  \begin{acknowledgements}
    I would love to thank my thesis advisor Benjamin Baumer, Assistant
    Professor of Statistical and Data Sciences at Smith College, for
    encouraging me to challenge myself and guiding me through this project.
    I want to thank Jordan Crouser for being my second reader and help me to
    revise my paper. I also want to thank all my friends and my roommates
    for their love and support.
  \end{acknowledgements}

  \hypersetup{linkcolor=black}
  \setcounter{tocdepth}{2}
  \tableofcontents

  \listoftables

  \listoffigures
  \begin{abstract}
    Yellow Taxi Cabs are widely recognized as the icons of New York City.
    The New York City Taxi \& Limousine Commission (TLC) provides publicly
    accessible yellow and green taxi trip records (N. T. staff, 2009a). Each
    taxi trip record is like a little piece of a gigantic puzzle, and all
    together they tell a story of what happens in New York City. This thesis
    presents a more efficient and easy-to-use way for users to retrieve trip
    records information of both New York City taxi and other ride-sharing
    services, such as Uber and Lyft, in New York City. By analyzing trip
    records of New York City's iconic yellow taxi, we seek answers to
    questions that are commonly asked by taxi drivers, passengers, and TLC
    officials to help all three parties to improve their services or
    experiences.
  \end{abstract}

\mainmatter % here the regular arabic numbering starts
\pagestyle{fancyplain} % turns page numbering back on

\chapter{Introduction}\label{introduction}

\section{Motivation}\label{motivation}

When is the best time during a day to travel to JKF Airport from
Brooklyn? How much tip do passengers usually pay to the taxi drivers? Is
the \$52 flat rate from Manhattan to JFK Airport appropriate? Questions
about New York City taxicabs are frequently asked by people travelling
in taxis in New York City. New York City Taxi and Limousine Commission
(TLC) provides taxi trip data on
\href{http://www.nyc.gov/html/tlc/html/about/trip_record_data.shtml}{their
website} for people to study and answer these questions. However, it is
not easy to work with taxi trip data provided by TLC, because there are
more than 250,000 taxi trips happenning everyday in New York City
(Whitford, 2017), which implies the large size of the datasets.

Working with medium data, such as the taxi TLC trips records, in
\textbf{R} is not an easy task. Loading medium-sized data into the
\textbf{R} environment takes a long time and might crash an \textbf{R}
session. Creating a user-friendly platform that allows \textbf{R} users
to easily work with medium data is my motivation. In my study, I focus
on New York City taxicab data because there are a lot of interesting
questions about New York City taxicabs that I want to explore.

New York City taxi drivers, passengers, and New York City TLC are the
three parties who are closely involved in the New York City taxi
industry. Each party has its own needs. Better understanding the needs
of the three parties and providing solutions to satisfy their needs is
the goal of this thesis.

This work contains two main components. The first component is building
the tool to work with the TLC taxi trip data, and the second component
is using the tool we build to understand the taxicab and e-hail service
use in New York City.

\section{Background}\label{background}

\subsection{Yellow Taxi}\label{yellow-taxi}

NYC Taxicabs are operated by private firms and licensed by the New York
City Taxi and Limousine Commission (TLC). TLC issues medallions to
taxicabs, and every taxicab must have a medallion to operate. There were
13,437 yellow medallion taxicabs licenses in 2014, and taxi patronage
has declined since 2011 because of the competition caused by rideshare
services (W. staff, 2018a).

\subsection{Green Taxi}\label{green-taxi}

The apple green taxicabs in New York City are called Boro taxis and they
are only allowed to pick up passengers in the outer boroughs and in
Manhattan above East 96th and West 110th Streets. Historically, only the
yellow medallion taxicabs were allowed to pick up passengers on the
street. However, since 95\% of yellow taxi pick-ups occurred in
Manhattan to the South of 96th Street and at the two airports, the Five
Borough Taxi Plan was started to allow green taxis to fill in the gap in
outer boroughs in the summer of 2013 (N. T. staff, 2009d).

\subsection{Uber}\label{uber}

Uber Technologies Inc. is an American technology company that operates
private cars worldwide. Uber drivers use their own cars, instead of
corporate-owned vehicles, to drive with Uber. In NYC, Uber uses `upfront
pricing', meaning that riders are informed about the fares that they
will pay before requesting a ride, and gratuity is not required. Riders
are given the opportunity to compare different transportation fares
before making their decisions on which one to choose. Uber NYC was
launched in May 2011 (Griswold, 2015).

\subsection{Lyft}\label{lyft}

Similar to Uber, Lyft is also an on-demand transportation company, and
it operates the Lyft car transportation mobile app. Lyft is the main
competitor of Uber, and it came into market in July 2014 in New York
City (Griswold, 2015).

\section{Literature Review}\label{literature-review}

\subsection{New York City Traffic and
Taxi}\label{new-york-city-traffic-and-taxi}

New York City is one of the most popular cities in the United States,
and New York City taxicabs represent the image of New York City. New
York City's traffic is a popular topic in journalism, and different
aspects of it has been studied by journalists, such as Patricia Reaney.
New York City's traffic is a nightmare, and the city officials have long
been trying to solve the congestion problem. In 2009, New York City was
voted to be the U.S. city with the ``angriest and most aggressive
drivers''. (Reaney, 2009) The bad temper of drivers are exacerbated by
New York City's severe cogestion.

How bad is the cogestion? In a journal published by New York Post in
2016, New York City was described as ``the city that never moves''.
(Danielle Furfaro \& Fears, 2016) What has led to the congestion in the
city? A journal from New York Post tried to find an answer to this
question: According to a former top NYPD official, ``The city streets
are being engineered to create traffic congestion, to slow traffic down,
to favor bikers and pedestrians'' so that drivers will have the
incentive to leave their cars at home and turn to mass transit or
bicycles (Sugar, 2017).

No matter how miserable the driving experiences are, taxi drivers have
no luxury to choose alternative transportation, and instead thay have to
consistently drive their cabs, which are usually surrounded by bad
traffic, in order to make a living.

\subsection{Competition between New York City taxi and e-hail
services}\label{competition-between-new-york-city-taxi-and-e-hail-services}
\begin{figure}

{\centering \includegraphics[width=5.33in]{figure/totals_by_car_type} 

}

\caption{NYC Monthly Taxi Pickups}\label{fig:totals-by-car-type}
\end{figure}
As shown in the visulization above (Schneider, 2015), the number of New
York City yellow taxi trips has been consistently declining for about 4
years, and the numbers of Uber and Lyft trips keep increasing. In 2017,
for the first time, the total number of monthly Uber trips exceeded the
number of yellow taxi trips.

Some studies have shown how competitive Uber and Lyft are. In 2017, Uber
and Lyft registered vehicles outnumbered NYC yellow cabs by 4 to 1.
(Sugar, 2017) Even though Yellow cab used to be the most popular
street-hail transportation service in New York City, passengers nowadays
tend to choose the more convenient options, ride-hailing apps.(Hu, 2017)

As reported in a jounal from Forbes Tech, data scientists from the
University of Cambridge in the UK and the University of Namur in Belgium
found that yellow taxi rides are on average \$1.40 cheaper than Uber X,
which is one type of economy ride service offered by Uber (H. staff,
2018). Moreover, uber appears more expensive for trips that are cheaper
than \$35, and less expensive than yellow taxi ride for trips that are
more expensive than \$35. Therefore, for short trips, taking a taxi is
more affordable. (Guerrini, 2015)

Apps, such as Openstreetcab, that compares the price of Uber and taxi
trips are designed to help customers to compare the fares of different
transportations. (O. staff, 2015)

\subsection{etl R package}\label{etl-r-package}

In
\texttt{R\ Markdown:\ Integrating\ A\ Reproducible\ Analysis\ Tool\ into\ Introductory\ Statistics},
the authors have presented experimental and statistical evidence that
\emph{R Markdown} replaced the antiquated and hard-to-reproduce
\texttt{copy-and-paste\ workflow}, and makes creating fully-reprodicible
statistical analysis straight-forward (B. Baumer, Cetinkaya-Rundel,
Bray, Loi, \& Horton, 2014).

Work with taxi trip data is not an easy task, because of the large size
of the taxi trip datasets (Whitford, 2017). Taxi trip datasets are
alssified as `medium data'. Loading medium-sized data into \textbf{R}
environment takes a long time and might crush an \textbf{R} session.
\texttt{etl} \textbf{R} package creates a user-friendly platform that
allows \textbf{R} users to easily work with medium data with the
extract, transform, load framework, which is commonly konwn as ETL in
computing (W. staff, 2018b). The \texttt{ETL} process has been set up
(B. S. Baumer, 2017) in \textbf{R} to facilitate etl operations for
medium data, and it is designed to work with any general data set.
Packages that are specific to perticular data sets are needed to be
written in order to better work with complex medium-sized data sets.

\section{Contribution}\label{contribution}

\subsection{\texorpdfstring{`nyctaxi'
Package}{nyctaxi Package}}\label{nyctaxi-package}

\texttt{nyctaxi} is an \textbf{R} package that help users to easily get
access to New York City Taxi, Uber and Lyft trip data through Extract,
Transform, and Load functions (ETL). (B. S. Baumer, 2017) This package
facilitates ETL to deal with medium data that are too big to store in
memory on a laptop. Users are given the option to choose specific years
and months as the input parameters of the three ETL functions, and a
connection to a populated SQL database will be returned as the output.
Users do not need to learn SQL queries, since all user interaction is in
\textbf{R}.
\begin{center}\includegraphics[width=5.88in]{figure/nyctaxi-page} \end{center}

\subsection{Social Impact of NYC Taxi}\label{social-impact-of-nyc-taxi}

NYC Taxi drivers wants to make the most profit. Taxi passengers want the
cheapest and most convenient way of transportantion. Since Uber and Lyft
launched their services in New York City, many consumers started to
demand the cheaper e-hail services (Sugar, 2017). TLC wants to protect
both taxi drivers and passengers, and it creates policies to make NYC
taxi more accessible to consumers who really need this service. In these
sections, we analyze what each party wants and try to find a way for
them to be achieve their goals.

\subsection{Reproducible Research}\label{reproducible-research}

Reproducible research and open source are two main emphasis of this
honors project. As scholars place more emphasis on the reproducibility
of research studies, it is essential for us to make our data and code
openly available for people to redo the analysis.

\texttt{Knitr} and Github are used in my project to make my study
reproducible, ranging from the initial source to raw data to the package
I wrote to utlize the raw data to the statistical data analysis. I used
an \textbf{R} package called \texttt{thesisdown} to typeset this paper,
this tool allows authors to create reproducible and dynamic techinical
report in \textbf{R} Markdown. It also allows users to embed \textbf{R}
code and interactive applicationis, and output into PDF, Word, ePub, or
gitbook doocuments. \texttt{thesisdown} helps users to efficiently put
together any paper with similar format.

Github is used to store the scripts for \texttt{nyctaxi} and this
thesis. \texttt{nyctaxi} is available on CRAN for people to download and
install (W. P. Li, Baumer, \& Trang Le, 2017), and the source code for
data analysis in this thesis is available under the Github account of
the author so that scholars can easily access the information that thery
are interested in. In terms of tables, figures, and anything included in
the Appendix attached to this thesis, scripts that are used to generate
them are included in the Github repository.

\chapter{Data and nyctaxi Package}\label{chapter2}

\section{Data and Storage}\label{data-and-storage}

The \texttt{nyctaxi} \textbf{R} package allows users to download, clean,
and load data into SQL databases. There are four types of data that are
available for users to get access to: trip level yellow taxi data from
2009 to the most recent month, trip level green taxi data from August
2013 to the most recent month, Uber pick-up data from April to September
2014 and from Janaury to June 2015, and weekly-aggregated Lyft trip data
from 2016 to the most recent week (W. P. Li et al., 2017).

\subsection{Yellow Taxi}\label{yellow-taxi-1}

The total size of all yellow taxi trip data \texttt{csv} files (from Jan
2010 to Dec 2016) is 191.38 GB, and NYC yellow taxi trip data from Jan
2009 to the most recent month can be found on the NYC Taxi \& Limousine
Commission (TLC) website (N. T. staff, 2009b). The data were collected
and provided to the NYC TLC by technology providers authorized under the
Taxicab \& Livery Passenger Enhancement Programs (TPEP/LPEP).

The yellow taxi trip records include the following fields: pick-up and
drop-off dates/times, pick-up and drop-off locations, trip distances,
itemized fares, rate types, payment types, and driver-reported passenger
counts.

\subsection{Green Taxi}\label{green-taxi-1}

The total size of green taxi trip data \texttt{csv} files (from Aug 2013
to Dec 2016) is 7.8 GB, and green taxi trip data from Aug 2013 to the
most recent month can be downloaded from NYC Taxi \& Limousine
Commission (TLC). (N. T. staff, 2009b) Green taxi trip records include
the same variables as yellow taxi trip records.

\subsection{TLC Summary Report}\label{tlc-summary-report}

The New York City TLC publishes summary reports that include aggregate
statistics about taxi, Uber, and Lyft usage. These are in addition to
the trip-level data; although the summary reports contain much less
detail, they're updated more frequently, which provides a more current
glimpse into the state of the cutthroat NYC taxi market. (N. T. staff,
2009a)

In addition, the trip level NYC Uber data only covers two periods, from
April to September 2014 and from January to June 2015. However, the
summary reports cover weekly-aggreagated data from 2015 to the most
recent week.
\begin{center}\includegraphics[width=5.84in]{figure/a-report} \end{center}

The data can be accessed by using the following commands:
\begin{itemize}
\tightlist
\item
  Yellow taxi data
\end{itemize}
\begin{Shaded}
\begin{Highlighting}[]
\KeywordTok{download.file}\NormalTok{(}\StringTok{"http://www.nyc.gov/html/tlc/downloads/csv/data_reports_monthly_indicators_yellow.csv"}\NormalTok{, }\DataTypeTok{destfile =} \StringTok{"~/Desktop/yellow_monthly_data.csv"}\NormalTok{)}
\end{Highlighting}
\end{Shaded}
*Uber and Lyft data
\begin{Shaded}
\begin{Highlighting}[]
\KeywordTok{download.file}\NormalTok{(}\StringTok{"http://data.cityofnewyork.us/api/views/2v9c-2k7f/rows.csv?accessType=DOWNLOAD"}\NormalTok{, }\DataTypeTok{destfile =} \StringTok{"~/Desktop/fhv_weekly_data.csv"}\NormalTok{)}
\end{Highlighting}
\end{Shaded}
\subsection{Uber}\label{uber-1}

The total size of Uber pick-up data (from Apr to Sep 2014 and from Jan
to June 2015) is 900 MB, and thanks to FiveThirtyEight who obtained the
data from NYC TLC by submitting a Freedom of Information Law (FOIL)
request on July 20, 2015, these data are now open to public (N. T.
staff, 2009c).

The 2014 Uber data contains 4 variables: \texttt{Date/Time} (the date
and time of the Uber pick-up), \texttt{Lat} (the latitude of the Uber
pick-up), \texttt{Lon} (the longitude of the Uber pick-up), and
\texttt{Base} (the TLC base company code affiliated with the Uber
pickup).

The 2015 Uber data contains 4 variables: \texttt{Dispatching\_base\_num}
(the TLC base company code of the base that dispatched the Uber),
\texttt{Pickup\_date} (the date of the Uber pick-up),
\texttt{Affiliated\_base\_num} (the TLC base company code affiliated
with the Uber pickup), and \texttt{locationID} (the pick-up location ID
affiliated with the Uber pickup).

NYC Open Data also provides weekly-aggreagated Uber pick-up data from
2015 to the most recent month. (N. O. staff, 2015b)

\subsection{Lyft}\label{lyft-1}

The total size of weekly-aggregated Lyft trip data (from Jan 2015 to Dec
2016) is 914.9 MB, and these data are open to public and
weekly-aggregated Lyft data from 2015 to the most recent week can be
found on NYC OpenData website. (N. O. staff, 2015a)

\subsection{Data Storage}\label{data-storage}

The total size of all \texttt{csv} files of the four services is about
200 GB, and a laptop usually has memory less than or equal to 8 GB.
Limited memory constrains the amount of data that can be loaded by a
personal computer at one time. When users load data into \textbf{R}
environment, \textbf{R} keeps them in memory; when the amount of data
loaded into \textbf{R} environment gets close to the limit of a
computer's memory, \textbf{R} becomes unresponsive or force quit the
current session. Therefore, better ways to work with data that takes
more space than 8 GB is needed. According to Zhang (2016), comparing to
RAM, hard disk is often used to store medium-sized data, because it is
affordable and are designed for storing large items permanently.
However, retrieving data from hard drives is about 1,000,000 times
slower.

\section{ETL nyctaxi Package}\label{etl-nyctaxi-package}

\texttt{etl} is the parent package of \texttt{nyctaxi}. \texttt{etl}
provides a framework that allows \textbf{R} users to work with medium
data without any knowledge in SQL database. Users can run SQL queries by
using \texttt{dplyr} commands in \textbf{R} and choose to only return
the final result, which could be a summary table, from SQL database into
\textbf{R} Environment in order to aovid \textbf{R} from crashing. The
user interaction takes place solely within \textbf{R}.

\texttt{etl} framework has three operations -Extract, Transfer, and
Load- which bring real-time data into local or remote SQL databases.
Users can specify which type of SQL database they prefer to connect to.
\texttt{etl}-dependent packages, such as \texttt{nyctaxi}, make medium
data more accessible to a wider audience. (B. Baumer et al., 2014)

\texttt{nyctaxi} was initially designed to work with New York City taxi
data, but later on Uber and Lyft data were added and the ETL functions
are modified to be specialized in working with these data. This package
compiles three major sources of hail service in New York City so that it
is convenient for users to compare and contrast the performance of these
three services. (W. P. Li et al., 2017)

This package inherits functions from many packages: \texttt{etl},
\texttt{dplyr}, \texttt{DBI}, \texttt{rlang}, and \texttt{stringr}.

Since SQL databases are good tools for medium data analysis, ETL
functions build connection to a SQL database at the back end and convert
\textbf{R} code automatically into SQL queries and send them to the SQL
database to get data tables containing data of each hail service. Thus,
users do not need to have any knowledge of SQL queries and they can draw
in any subsets of the data from the SQL database in \textbf{R}.

In general, \texttt{etl\_extract.etl\_nyctaxi()} function download data
of the four types of hail service data (yellow taxi, green taxi, Uber,
and Lyft) from the corresponding sources.
\texttt{etl\_transform.etl\_nyctaxi()} uses different techniques to
clean all four types of data to get then ready for the next step.
\texttt{etl\_load.etl\_nyctaxi()} loads the data user selected to a SQL
database.

The Comprehensive \textbf{R} Archive Network (CRAN) is a collection of
sites that carry identical material, consisting of the \textbf{R}
distributions, the contributed extensions, documentation for \textbf{R},
and binaries. (C. staff, n.d.) \texttt{nyctaxi} \textbf{R} package lives
on CRAN, and it can be installed with the \texttt{install.packages()}
function in \textbf{R}.
\begin{Shaded}
\begin{Highlighting}[]
\KeywordTok{install.packages}\NormalTok{(}\StringTok{"nyctaxi"}\NormalTok{)}
\end{Highlighting}
\end{Shaded}
\begin{center}\includegraphics[width=5.88in]{figure/nyctaxi-page} \end{center}

Users need to create an \texttt{etl} object in order to apply the etl
operations to it, and only the name of the SQL database, working
directory, and type of SQL database need to be specified during
initialization. If the type of SQL database is not specified, a local
RSQLite database will be generated as default.
\begin{Shaded}
\begin{Highlighting}[]
\NormalTok{db <-}\StringTok{ }\KeywordTok{src_mysql}\NormalTok{(}\StringTok{"nyctaxi"}\NormalTok{, }\DataTypeTok{user =} \StringTok{"urname"}\NormalTok{, }\DataTypeTok{host =} \StringTok{"host"}\NormalTok{, }\DataTypeTok{password =} \StringTok{"pw"}\NormalTok{)}
\NormalTok{taxi <-}\StringTok{ }\KeywordTok{etl}\NormalTok{(}\StringTok{"nyctaxi"}\NormalTok{, }\DataTypeTok{dir =} \StringTok{"~/Desktop/nyctaxi"}\NormalTok{, db)}
\end{Highlighting}
\end{Shaded}
In the example above, a folder called \texttt{nyctaxi} is created on the
desktop and a connection to a MySQL database is generated. In the
procession of initialization, a local folder contains two subfolders,
\texttt{raw} and \texttt{load}, are also created under the directory the
user specifies. \texttt{raw} folder stores data downloaded from online
open sources, and \texttt{load} folder stores cleaned CSV data files
that are ready to be loaded into SQL database. The ETL framework keeps
data directly scraped from online data sources in their original forms.
In this way, the original data is always available to users in case data
corruption happens in later stages.

After an etl object is created (nyctaxi is the etl object in this case),
four parameters are needed to specify the data that users want: (1)
\texttt{obj}: an etl object; (2) \texttt{years}: a numeric vector giving
the years, and the default is the most recent year; (3) \texttt{months}:
a numeric vector giving the months, and the default is \texttt{1:12};
(4) \texttt{type}: a character variable giving the type of data the user
wants to download. There are four types: \texttt{yellow},
\texttt{green}, \texttt{uber}, and \texttt{lyft}. The default is
\texttt{yellow}.

\subsection{\texorpdfstring{Taxi zone shapefile attached to nyctaxi
\textbf{R}
package}{Taxi zone shapefile attached to nyctaxi R package}}\label{taxi-zone-shapefile-attached-to-nyctaxi-r-package}

Two datasets are attached to \texttt{nyctaxi}. The first one is called
\texttt{taxi\_zone\_lookup}, and this dataset contains information, such
as taxi zone location IDs, location names, and corresponding boroughs
for each ID. (N. T. staff, 2009b) A shapefile containing the boundaries
for the taxi zones, \texttt{taxi\_zones}, is also included in the
package for users to do spatial analysis. Visulizations similar to one
shown below can be generated with the shapefile.
\begin{figure}

{\centering \includegraphics[width=5.84in]{figure/zonemap} 

}

\caption{NYC Taxi Zone Map}\label{fig:zonemap}
\end{figure}
\section{Extract-Transform-Load}\label{extract-transform-load}

\subsection{Extract}\label{extract}

\texttt{etl\_extract.etl\_nyctaxi()} allows users to download New York
City yellow taxi, green taxi, Uber, and Lyft data from the corresponding
data sources. It takes the \texttt{years}, \texttt{months}, and
\texttt{type} parameters and download the New York City taxi data
specified by users. New York City Yellow and Green Taxi data are updated
on NYC Taxi \& Limousine Commission (TLC) website on a monthly basis.
\begin{Shaded}
\begin{Highlighting}[]
\NormalTok{taxi }\OperatorTok\StringTok{ }
\StringTok{   }\KeywordTok{etl_extract}\NormalTok{(}\DataTypeTok{years =} \DecValTok{2014}\OperatorTok{:}\DecValTok{2016}\NormalTok{, }\DataTypeTok{months =} \DecValTok{1}\OperatorTok{:}\DecValTok{12}\NormalTok{, }\DataTypeTok{type =} \KeywordTok{c}\NormalTok{(}\StringTok{"yellow"}\NormalTok{, }\StringTok{"green"}\NormalTok{))}
\end{Highlighting}
\end{Shaded}
Uber trip record data is static and small, so we decided to only give
users the options to either download all data from April to Sepetember,
2014 or download all Uber trip records from Janaury to June, 2015 at
onc. Users do not have the ability to download Uber data from a specific
month.
\begin{Shaded}
\begin{Highlighting}[]
\NormalTok{taxi }\OperatorTok\StringTok{ }
\StringTok{   }\KeywordTok{etl_extract}\NormalTok{(}\DataTypeTok{years =} \DecValTok{2014}\OperatorTok{:}\DecValTok{2016}\NormalTok{, }\DataTypeTok{months =} \DecValTok{1}\OperatorTok{:}\DecValTok{12}\NormalTok{, }\DataTypeTok{type =} \KeywordTok{c}\NormalTok{(}\StringTok{"uber"}\NormalTok{))}
\end{Highlighting}
\end{Shaded}
Lyft data is updated on NYC Open Data webiste on a weekly basis. Since
the weekly-aggregated data is tiny and only data later then 2014 is
available, we decided to only allow users to download Lyft data by year.
\begin{Shaded}
\begin{Highlighting}[]
\NormalTok{taxi }\OperatorTok\StringTok{ }
\StringTok{   }\KeywordTok{etl_extract}\NormalTok{(}\DataTypeTok{years =} \DecValTok{2014}\OperatorTok{:}\DecValTok{2016}\NormalTok{, }\DataTypeTok{months =} \DecValTok{1}\OperatorTok{:}\DecValTok{12}\NormalTok{, }\DataTypeTok{type =} \KeywordTok{c}\NormalTok{(}\StringTok{"lyft"}\NormalTok{))}
\end{Highlighting}
\end{Shaded}
The default \texttt{years} is the current year, and the default
\texttt{months} are the all twelve months. The default type of
transportation is \texttt{yellow}. When an invalid month is entered,
warning message will suggest users to reconsider their choice and select
a new set of month.

An utility function, \texttt{download\_nyc\_data()}, was written to be
used in \texttt{etl\_extract.etl\_nyctaxi()} to make this function more
condense (Appendix A).

\subsection{Transform}\label{transform}

\texttt{etl\_transform.etl\_nyctaxi()} allows users to transform New
York City yellow taxi, green taxi, Uber, and Lyft data into cleaned
formats, and it utlizes different data cleaning techniques when it
transforms data for each transportation type. In general, it cleans the
data and creates a new \texttt{csv} file in the \texttt{load} directory
to store the cleaned data. It helps us to retain and protect raw data
from being modified or destroyed. Users are allowed to specify the month
of interest in order to only transform the data that they are interested
in. This functionality helps people to be more efficient with their use
of time.

By default, it takes the current year Yellow taxi trip records data
files, and save copies of them in the \texttt{load} diectory. It skips
the cleaning step, because the raw Yellow Taxi data downloaded from TLC
is already in a desired format with all variables correctly labelled.
\begin{Shaded}
\begin{Highlighting}[]
\NormalTok{taxi }\OperatorTok\StringTok{ }
\StringTok{   }\KeywordTok{etl_transform}\NormalTok{(}\DataTypeTok{years =} \DecValTok{2014}\OperatorTok{:}\DecValTok{2016}\NormalTok{, }\DataTypeTok{months =} \DecValTok{1}\OperatorTok{:}\DecValTok{12}\NormalTok{, }\DataTypeTok{type =} \KeywordTok{c}\NormalTok{(}\StringTok{"yellow"}\NormalTok{, }\StringTok{"green"}\NormalTok{, }\StringTok{"uber"}\NormalTok{, }\StringTok{"lyft"}\NormalTok{))}
\end{Highlighting}
\end{Shaded}
There are a few main transformations that are done by this function:

\subsubsection{Green Taxi -- Extra Blank Row and
Column}\label{green-taxi-extra-blank-row-and-column}

Green Taxi monthly data from August 2013 to the most recent month
besides 2015 all have a blank second row in the \texttt{csv} files.
Similar to this problem, Green Taxi data from 2013, 2014, and 2015 all
have an extra blank columns attanched to the right-most column. These
blank rows and columns cause problems in the later stage when users want
to load data into SQL database. In order to get Green Taxi data ready
for the \texttt{load} phase, we used the \texttt{system()} function in
\textbf{R} to invoke the Terminal command specified to remove the blank
rows and columns.

\subsubsection{Uber Data -- Reconciling Inconsistent
Filenames}\label{uber-data-reconciling-inconsistent-filenames}

Uber only released over 4.5 million data records from April to September
2014 and 14.3 million records from Janaury to June 2015. Information of
different sets of variables are released for 2014 and 2015, and
variables have different naming convention. When users want to download
data from both years, variables are renamed so that data from both years
can be cosolidated into one big dataset with consistent variable names.

\subsubsection{Uber Data -- Reconciling Inconsistent Data
Formats}\label{uber-data-reconciling-inconsistent-data-formats}

The data type of \texttt{Date/Time} variable in Uber datasets is
originally encoded as \texttt{character}. In order to enable it to be
recognized as \texttt{timestamp} by \textbf{R}, we use \texttt{ymd\_hms}
in \texttt{lubridate} (Grolemund \& Wickham, 2011) to transform date
time to \texttt{POSIXct} objects.

\subsubsection{Optimizing I/O Process}\label{optimizing-io-process}

Improving file input and output processes is an important part of
\texttt{etl\_transform}. \texttt{data.table} (Dowle \& Srinivasan, 2017)
only takes half of the time to read from and write into datasets
comparing to \texttt{readr} (Wickham, Hester, \& Francois, 2017).
Therefore, \texttt{etl\_transform} uses \texttt{fread()} and
\texttt{fwrite()} from \texttt{data.table} instead of \texttt{read\_csv}
or \texttt{write\_csv} from \texttt{readr} to reduce the data processing
time (Zhang, 2017).

\subsection{Load}\label{load}

\texttt{etl\_load.etl\_nyctaxi()} allows users to load New York City
yellow taxi, green taxi, Uber, and Lyft data into different data tables
in a SQL database. It populates a SQL database with data cleaned by
\texttt{etl\_transform}.
\begin{Shaded}
\begin{Highlighting}[]
\NormalTok{taxi }\OperatorTok\StringTok{ }
\StringTok{   }\KeywordTok{etl_load}\NormalTok{(}\DataTypeTok{years =} \DecValTok{2014}\OperatorTok{:}\DecValTok{2016}\NormalTok{, }\DataTypeTok{months =} \DecValTok{1}\OperatorTok{:}\DecValTok{12}\NormalTok{, }\DataTypeTok{type =} \KeywordTok{c}\NormalTok{(}\StringTok{"yellow"}\NormalTok{, }\StringTok{"green"}\NormalTok{, }\StringTok{"uber"}\NormalTok{, }\StringTok{"lyft"}\NormalTok{))}
\end{Highlighting}
\end{Shaded}
\subsection{SQL Database
Initialization}\label{sql-database-initialization}

\texttt{init.mysql()} is written under \texttt{nyctaxi} to help users to
set up five basic table structures for MySQL database.
\texttt{yellow\_old} is created for Yellow Taxi data that are prior to
August 2016, and \texttt{yellow} is created for data later than July
2016. \texttt{green}, \texttt{uber}, and \texttt{lyft} are also
initiated for the three transportations.

\texttt{etl\_init()} can be run after a database connection is built to
process to process \texttt{init.mysql()} to initialize a MySQL database,
and default columns with the correct variable names and typed defined
will be automatically generated.
\begin{Shaded}
\begin{Highlighting}[]
\NormalTok{taxi }\OperatorTok
\StringTok{  }\KeywordTok{etl_init}\NormalTok{()}
\end{Highlighting}
\end{Shaded}
In order to increase the query speed at the data analysis stage,
\texttt{KEY}s are created for multiple variables for each
transportation. Since there is no variable containing unique value for
each observation, no primary variable is needed. Using \texttt{KEY}s in
data analysis query can speed up the query process.

Due to the large size of Yellow Taxi datasets, \texttt{yellow\_old} and
\texttt{yellow} are partitioned into subgroups by \texttt{year}. When we
need to run a query on data from a specific year, having partitions
allows MySQL to directly find the data specified without filtering on
every single row. It speeds up the query process. A \texttt{VIEW} called
\texttt{yellow\_old\_sum} is also created to generate a summary table
for the number of Yellow Taxi trips in each month.
\begin{figure}

{\centering \includegraphics[width=5.76in]{figure/mysql_view} 

}

\caption{MySQL View}\label{fig:unnamed-chunk-21}
\end{figure}
\section{New York City Taxicab and E-hail Services
Summary}\label{new-york-city-taxicab-and-e-hail-services-summary}
\begin{figure}

{\centering \includegraphics[width=5.75in]{figure/Num_trips_summary} 

}

\caption{Summary of Number of trips Made by 4 Types of Transportations between 2014 and 2016 in NYC}\label{fig:num-trips-summary}
\end{figure}
Figure \ref{fig:num-trips-summary} is a summary of total number of trips
made by all 4 types of transporations that are available to users from
2014 to 2016. In order to generate this summary, I combined trip-level
yellow and green taxi data from TLC trip data website and weekly Uber
and Lyft data from NYC OpenData.Data used in Figure
\ref{fig:num-trips-summary} can be accessed by running the code below:
\begin{itemize}
\tightlist
\item
  Yellow taxi monthly data
\end{itemize}
\begin{Shaded}
\begin{Highlighting}[]
\KeywordTok{download.file}\NormalTok{(}\StringTok{"http://www.nyc.gov/html/tlc/downloads/csv/data_reports_monthly_indicators_yellow.csv"}\NormalTok{, }\DataTypeTok{destfile =} \StringTok{"~/Desktop/yellow_monthly_data.csv"}\NormalTok{)}
\end{Highlighting}
\end{Shaded}
\begin{itemize}
\tightlist
\item
  Uber weekly data
\end{itemize}
\begin{Shaded}
\begin{Highlighting}[]
\KeywordTok{download.file}\NormalTok{(}\StringTok{"https://data.cityofnewyork.us/resource/gt3n-7ri6.csv"}\NormalTok{, }\DataTypeTok{destfile =} \StringTok{"~/Desktop/uber_weekly_data.csv"}\NormalTok{)}
\end{Highlighting}
\end{Shaded}
\begin{itemize}
\tightlist
\item
  Lyft weekly data
\end{itemize}
\begin{Shaded}
\begin{Highlighting}[]
\KeywordTok{download.file}\NormalTok{(}\StringTok{"https://data.cityofnewyork.us/resource/juxc-sutg.csv"}\NormalTok{, }\DataTypeTok{destfile =} \StringTok{"~/Desktop/lyft_weekly_data.csv"}\NormalTok{)}
\end{Highlighting}
\end{Shaded}
\section{Source Code}\label{source-code}

\subsection{ETL Extract}\label{etl-extract}
\begin{Shaded}
\begin{Highlighting}[]
\NormalTok{etl_extract.etl_nyctaxi <-}\StringTok{ }
\StringTok{  }\ControlFlowTok{function}\NormalTok{(obj, }
\DataTypeTok{years =} \KeywordTok{as.numeric}\NormalTok{(}\KeywordTok{format}\NormalTok{(}\KeywordTok{Sys.Date}\NormalTok{(),}\StringTok{'%Y'}\NormalTok{)), }
                                    \DataTypeTok{months =} \DecValTok{1}\OperatorTok{:}\DecValTok{12}\NormalTok{, }
                                    \DataTypeTok{type  =} \StringTok{"yellow"}\NormalTok{,...) \{}
  \CommentTok{#TAXI YELLOW-------------------------}
\NormalTok{  taxi_yellow <-}\StringTok{ }\ControlFlowTok{function}\NormalTok{(obj, years, months,...) \{}
    \KeywordTok{message}\NormalTok{(}\StringTok{"Extracting raw yellow taxi data..."}\NormalTok{)}
\NormalTok{    remote <-}\StringTok{ }\NormalTok{etl}\OperatorTok{::}\KeywordTok{valid_year_month}\NormalTok{(years, months, }
    \DataTypeTok{begin =} \StringTok{"2009-01-01"}\NormalTok{) }\OperatorTok
\StringTok{      }\KeywordTok{mutate_}\NormalTok{(}\DataTypeTok{src =} 
  \OperatorTok{~}\KeywordTok{file.path}\NormalTok{(}\StringTok{"https://s3.amazonaws.com/nyc-tlc/trip+data"}\NormalTok{, }
                               \KeywordTok{paste0}\NormalTok{(}\StringTok{"yellow"}\NormalTok{, }\StringTok{"_tripdata_"}\NormalTok{, year, }\StringTok{"-"}\NormalTok{,}
\NormalTok{                      stringr}\OperatorTok{::}\KeywordTok{str_pad}\NormalTok{(month, }\DecValTok{2}\NormalTok{, }\StringTok{"left"}\NormalTok{, }\StringTok{"0"}\NormalTok{), }\StringTok{".csv"}\NormalTok{))) }
    \KeywordTok{tryCatch}\NormalTok{(}\DataTypeTok{expr =}\NormalTok{ etl}\OperatorTok{::}\KeywordTok{smart_download}\NormalTok{(obj, remote}\OperatorTok{$}\NormalTok{src, ...),}
             \DataTypeTok{error =} \ControlFlowTok{function}\NormalTok{(e)\{}\KeywordTok{warning}\NormalTok{(e)\}, }
             \DataTypeTok{finally =} \KeywordTok{warning}\NormalTok{(}\StringTok{"Only the following data are availabel on}
\StringTok{                                TLC: Yellow taxi data: 2009 Jan - }
\StringTok{                               last month"}\NormalTok{))\} }
  \CommentTok{#TAXI GREEN----------------------}
\NormalTok{  taxi_green <-}\StringTok{ }\ControlFlowTok{function}\NormalTok{(obj, years, months,...) \{}
    \KeywordTok{message}\NormalTok{(}\StringTok{"Extracting raw green taxi data..."}\NormalTok{)}
\NormalTok{    remote <-}\StringTok{ }\NormalTok{etl}\OperatorTok{::}\KeywordTok{valid_year_month}\NormalTok{(years, months, }\DataTypeTok{begin =} \StringTok{"2013-08-01"}\NormalTok{) }\OperatorTok
\StringTok{      }\KeywordTok{mutate_}\NormalTok{(}\DataTypeTok{src =} 
                \OperatorTok{~}\KeywordTok{file.path}\NormalTok{(}\StringTok{"https://s3.amazonaws.com/nyc-tlc/trip+data"}\NormalTok{, }
                               \KeywordTok{paste0}\NormalTok{(}\StringTok{"green"}\NormalTok{, }\StringTok{"_tripdata_"}\NormalTok{, year, }\StringTok{"-"}\NormalTok{,}
\NormalTok{                          stringr}\OperatorTok{::}\KeywordTok{str_pad}\NormalTok{(month, }\DecValTok{2}\NormalTok{, }\StringTok{"left"}\NormalTok{, }\StringTok{"0"}\NormalTok{), }\StringTok{".csv"}\NormalTok{)))}
    \KeywordTok{tryCatch}\NormalTok{(}\DataTypeTok{expr =}\NormalTok{ etl}\OperatorTok{::}\KeywordTok{smart_download}\NormalTok{(obj, remote}\OperatorTok{$}\NormalTok{src, ...),}
             \DataTypeTok{error =} \ControlFlowTok{function}\NormalTok{(e)\{}\KeywordTok{warning}\NormalTok{(e)\}, }
             \DataTypeTok{finally =} \KeywordTok{warning}\NormalTok{(}\StringTok{"Only the following data are availabel on TLC:}
\StringTok{                               Green taxi data: 2013 Aug - last month"}\NormalTok{))\} }
  \CommentTok{#UBER--------------------------------}
\NormalTok{  uber <-}\StringTok{ }\ControlFlowTok{function}\NormalTok{(obj, years, months,...) \{}
    \KeywordTok{message}\NormalTok{(}\StringTok{"Extracting raw uber data..."}\NormalTok{)}
\NormalTok{    raw_month_}\DecValTok{2014}\NormalTok{ <-}\StringTok{ }\NormalTok{etl}\OperatorTok{::}\KeywordTok{valid_year_month}\NormalTok{(}\DataTypeTok{years =} \DecValTok{2014}\NormalTok{, }\DataTypeTok{months =} \DecValTok{4}\OperatorTok{:}\DecValTok{9}\NormalTok{)}
\NormalTok{    raw_month_}\DecValTok{2015}\NormalTok{ <-}\StringTok{ }\NormalTok{etl}\OperatorTok{::}\KeywordTok{valid_year_month}\NormalTok{(}\DataTypeTok{years =} \DecValTok{2015}\NormalTok{, }\DataTypeTok{months =} \DecValTok{1}\OperatorTok{:}\DecValTok{6}\NormalTok{)}
\NormalTok{    raw_month <-}\StringTok{ }\KeywordTok{bind_rows}\NormalTok{(raw_month_}\DecValTok{2014}\NormalTok{, raw_month_}\DecValTok{2015}\NormalTok{)}
\NormalTok{    path =}\StringTok{ "https://raw.githubusercontent.com/}
\StringTok{    fivethirtyeight/uber-tlc-foil-response/master/uber-trip-data"}
\NormalTok{    remote <-}\StringTok{ }\NormalTok{etl}\OperatorTok{::}\KeywordTok{valid_year_month}\NormalTok{(years, months)}
\NormalTok{    remote_small <-}\StringTok{ }\KeywordTok{intersect}\NormalTok{(raw_month, remote)}
    \ControlFlowTok{if}\NormalTok{ (}\DecValTok{2015} \OperatorTok\StringTok{ }\NormalTok{remote_small}\OperatorTok{$}\NormalTok{year }\OperatorTok{&&}\StringTok{ }\OperatorTok{!}\NormalTok{(}\DecValTok{2014} \OperatorTok\StringTok{ }\NormalTok{remote_small}\OperatorTok{$}\NormalTok{year))\{}
      \CommentTok{#download 2015 data}
      \KeywordTok{message}\NormalTok{(}\StringTok{"Downloading Uber 2015 data..."}\NormalTok{)}
\NormalTok{      etl}\OperatorTok{::}\KeywordTok{smart_download}\NormalTok{(obj, }\StringTok{"https://github.com/fivethirtyeight/}
\StringTok{                          uber-tlc-foil-response/raw/master/}
\StringTok{                    uber-trip-data/uber-raw-data-janjune-15.csv.zip"}\NormalTok{,...)\}}
    \ControlFlowTok{else} \ControlFlowTok{if}\NormalTok{ (}\DecValTok{2015} \OperatorTok\StringTok{ }\NormalTok{remote_small}\OperatorTok{$}\NormalTok{year }\OperatorTok{&&}\StringTok{ }\DecValTok{2014} \OperatorTok\StringTok{ }\NormalTok{remote_small}\OperatorTok{$}\NormalTok{year) \{}
      \CommentTok{#download 2015 data}
      \KeywordTok{message}\NormalTok{(}\StringTok{"Downloading Uber 2015 data..."}\NormalTok{)}
\NormalTok{      etl}\OperatorTok{::}\KeywordTok{smart_download}\NormalTok{(obj, }\StringTok{"https://github.com/fivethirtyeight/}
\StringTok{                        uber-tlc-foil-response/raw/master/uber-trip-data}
\StringTok{                          /uber-raw-data-janjune-15.csv.zip"}\NormalTok{,...)}
      \CommentTok{#download 2014 data}
\NormalTok{      small <-}\StringTok{ }\NormalTok{remote_small }\OperatorTok
\StringTok{        }\KeywordTok{filter_}\NormalTok{(}\OperatorTok{~}\NormalTok{year }\OperatorTok{==}\StringTok{ }\DecValTok{2014}\NormalTok{) }\OperatorTok
\StringTok{        }\KeywordTok{mutate_}\NormalTok{(}\DataTypeTok{month_abb =} \OperatorTok{~}\KeywordTok{tolower}\NormalTok{(month.abb[month]),}
                \DataTypeTok{src =} \OperatorTok{~}\KeywordTok{file.path}\NormalTok{(path,}
                \KeywordTok{paste0}\NormalTok{(}\StringTok{"uber-raw-data-"}\NormalTok{,month_abb,}
                \KeywordTok{substr}\NormalTok{(year,}\DecValTok{3}\NormalTok{,}\DecValTok{4}\NormalTok{),}\StringTok{".csv"}\NormalTok{)))}
      \KeywordTok{message}\NormalTok{(}\StringTok{"Downloading Uber 2014 data..."}\NormalTok{)}
\NormalTok{      etl}\OperatorTok{::}\KeywordTok{smart_download}\NormalTok{(obj, small}\OperatorTok{$}\NormalTok{src,...) }
\NormalTok{    \} }\ControlFlowTok{else} \ControlFlowTok{if}\NormalTok{ (}\DecValTok{2014} \OperatorTok\StringTok{ }\NormalTok{remote_small}\OperatorTok{$}\NormalTok{year }\OperatorTok{&&}\StringTok{ }
\StringTok{    }\OperatorTok{!}\NormalTok{(}\DecValTok{2015} \OperatorTok\StringTok{ }\NormalTok{remote_small}\OperatorTok{$}\NormalTok{year)) \{}
      \KeywordTok{message}\NormalTok{(}\StringTok{"Downloading Uber 2014 data..."}\NormalTok{)}
      \CommentTok{#file paths}
\NormalTok{      small <-}\StringTok{ }\NormalTok{remote_small }\OperatorTok
\StringTok{        }\KeywordTok{mutate_}\NormalTok{(}\DataTypeTok{month_abb =} 
                  \OperatorTok{~}\KeywordTok{tolower}\NormalTok{(month.abb[month]),}
                \DataTypeTok{src =} \OperatorTok{~}\KeywordTok{file.path}\NormalTok{(path,}
                \KeywordTok{paste0}\NormalTok{(}\StringTok{"uber-raw-data-"}\NormalTok{,month_abb,}
                \KeywordTok{substr}\NormalTok{(year,}\DecValTok{3}\NormalTok{,}\DecValTok{4}\NormalTok{),}\StringTok{".csv"}\NormalTok{)))}
\NormalTok{      etl}\OperatorTok{::}\KeywordTok{smart_download}\NormalTok{(obj, small}\OperatorTok{$}\NormalTok{src,...)\}}
    \ControlFlowTok{else}\NormalTok{ \{}\KeywordTok{warning}\NormalTok{(}\StringTok{"The Uber data you requested are }
\StringTok{                  not currently available. Only data}
\StringTok{                  from 2014/04-2014/09 and 2015/01-}
\StringTok{                  2015/06 are available..."}\NormalTok{)\}}
\NormalTok{    \} }
  \CommentTok{#LYFT----------------------------------}
\NormalTok{  lyft <-}\StringTok{ }\ControlFlowTok{function}\NormalTok{(obj, years, months,...)\{}
    \KeywordTok{message}\NormalTok{(}\StringTok{"Extracting raw lyft data..."}\NormalTok{)}
    \CommentTok{#check if the week is valid}
\NormalTok{    valid_months <-}\StringTok{ }\NormalTok{etl}\OperatorTok{::}\KeywordTok{valid_year_month}\NormalTok{(years, months,}
    \DataTypeTok{begin =} \StringTok{"2015-01-01"}\NormalTok{)}
\NormalTok{    base_url =}\StringTok{ "https://data.cityofnewyork.us/}
\StringTok{    resource/edp9-qgv4.csv"}
\NormalTok{    valid_months <-}\StringTok{ }\NormalTok{valid_months }\OperatorTok
\StringTok{      }\KeywordTok{mutate_}\NormalTok{(}\DataTypeTok{new_filenames =} 
                \OperatorTok{~}\KeywordTok{paste0}\NormalTok{(}\StringTok{"lyft-"}\NormalTok{, year, }\StringTok{".csv"}\NormalTok{)) }\OperatorTok
\StringTok{      }\KeywordTok{mutate_}\NormalTok{(}\DataTypeTok{drop =} \OtherTok{TRUE}\NormalTok{)}
    \CommentTok{#only keep one data set per year}
\NormalTok{    year <-}\StringTok{ }\NormalTok{valid_months[}\DecValTok{1}\NormalTok{,}\DecValTok{1}\NormalTok{]}
\NormalTok{    n <-}\StringTok{ }\KeywordTok{nrow}\NormalTok{(valid_months)}
    \ControlFlowTok{for}\NormalTok{ (i }\ControlFlowTok{in} \DecValTok{2}\OperatorTok{:}\NormalTok{n) \{}
      \ControlFlowTok{if}\NormalTok{(year }\OperatorTok{==}\StringTok{ }\NormalTok{valid_months[i}\OperatorTok{-}\DecValTok{1}\NormalTok{,}\DecValTok{1}\NormalTok{]) \{}
\NormalTok{        valid_months[i,}\DecValTok{6}\NormalTok{] <-}\StringTok{ }\OtherTok{FALSE}
\NormalTok{        year <-}\StringTok{ }\NormalTok{valid_months[i}\OperatorTok{+}\DecValTok{1}\NormalTok{,}\DecValTok{1}\NormalTok{]}
\NormalTok{      \} }\ControlFlowTok{else}\NormalTok{ \{}
\NormalTok{        valid_months[i,}\DecValTok{6}\NormalTok{] <-}\StringTok{ }\OtherTok{TRUE}
\NormalTok{        year <-}\StringTok{ }\NormalTok{valid_months[i}\OperatorTok{+}\DecValTok{1}\NormalTok{,}\DecValTok{1}\NormalTok{]\}}
\NormalTok{      \}}
\NormalTok{    row_to_keep =}\StringTok{ }\NormalTok{valid_months}\OperatorTok{$}\NormalTok{drop}
\NormalTok{    valid_months <-}\StringTok{ }\NormalTok{valid_months[row_to_keep,]}
    
    \CommentTok{#download lyft files, try two different methods}
\NormalTok{    first_try<-}\KeywordTok{tryCatch}\NormalTok{(}
      \KeywordTok{download_nyc_data}\NormalTok{(obj, base_url, valid_months}\OperatorTok{$}\NormalTok{year, }
      \DataTypeTok{n =} \DecValTok{50000}\NormalTok{, }\DataTypeTok{names =}\NormalTok{ valid_months}\OperatorTok{$}\NormalTok{new_filenames),}
      \DataTypeTok{error =} \ControlFlowTok{function}\NormalTok{(e)\{}\KeywordTok{warning}\NormalTok{(e)\},}
      \DataTypeTok{finally =} \StringTok{'method = "libcurl" fails'}\NormalTok{)}
\NormalTok{  \}}
  
  \ControlFlowTok{if}\NormalTok{ (type }\OperatorTok{==}\StringTok{ "yellow"}\NormalTok{)\{}\KeywordTok{taxi_yellow}\NormalTok{(obj, years, months,...)\} }
  \ControlFlowTok{else} \ControlFlowTok{if}\NormalTok{ (type }\OperatorTok{==}\StringTok{ "green"}\NormalTok{)\{}\KeywordTok{taxi_green}\NormalTok{(obj, years, months,...)\}}
  \ControlFlowTok{else} \ControlFlowTok{if}\NormalTok{ (type }\OperatorTok{==}\StringTok{ "uber"}\NormalTok{)\{}\KeywordTok{uber}\NormalTok{(obj, years, months,...)\}}
  \ControlFlowTok{else} \ControlFlowTok{if}\NormalTok{ (type }\OperatorTok{==}\StringTok{ "lyft"}\NormalTok{)\{}\KeywordTok{lyft}\NormalTok{(obj, years, months,...)\}}
  \ControlFlowTok{else}\NormalTok{ \{}\KeywordTok{message}\NormalTok{(}\StringTok{"The type you chose does not exit..."}\NormalTok{)\}}
  
  \KeywordTok{invisible}\NormalTok{(obj)}
\NormalTok{\}}
\end{Highlighting}
\end{Shaded}
\subsection{ETL Transform}\label{etl-transform}
\begin{Shaded}
\begin{Highlighting}[]
\NormalTok{opts_chunk}\OperatorTok{$}\KeywordTok{set}\NormalTok{(}\DataTypeTok{tidy.opts=}\KeywordTok{list}\NormalTok{(}\DataTypeTok{width.cutoff=}\DecValTok{60}\NormalTok{))}
\NormalTok{etl_transform.etl_nyctaxi <-}\StringTok{ }\ControlFlowTok{function}\NormalTok{(obj, }
                          \DataTypeTok{years =} \KeywordTok{as.numeric}\NormalTok{(}\KeywordTok{format}\NormalTok{(}\KeywordTok{Sys.Date}\NormalTok{(),}\StringTok{'%Y'}\NormalTok{)), }
                          \DataTypeTok{months =} \DecValTok{1}\OperatorTok{:}\DecValTok{12}\NormalTok{, }
                          \DataTypeTok{type  =} \StringTok{"yellow"}\NormalTok{,...) \{}
  \CommentTok{#TAXI YELLOW-----------------------------}
\NormalTok{  taxi_yellow <-}\StringTok{ }\ControlFlowTok{function}\NormalTok{(obj, years, months) \{}
    \KeywordTok{message}\NormalTok{(}\StringTok{"Transforming yellow taxi data from raw to }
\StringTok{            load directory..."}\NormalTok{)}
    \CommentTok{#create a df of file path of the files that the user wants to transform}
\NormalTok{    remote <-}\StringTok{ }\NormalTok{etl}\OperatorTok{::}\KeywordTok{valid_year_month}\NormalTok{(years, months, }
    \DataTypeTok{begin =} \StringTok{"2009-01-01"}\NormalTok{) }\OperatorTok
\StringTok{      }\KeywordTok{mutate_}\NormalTok{(}\DataTypeTok{src =} \OperatorTok{~}\KeywordTok{file.path}\NormalTok{(}\KeywordTok{attr}\NormalTok{(obj, }\StringTok{"raw_dir"}\NormalTok{), }
      \KeywordTok{paste0}\NormalTok{(}\StringTok{"yellow"}\NormalTok{, }\StringTok{"_tripdata_"}\NormalTok{, year, }\StringTok{"-"}\NormalTok{,}
\NormalTok{      stringr}\OperatorTok{::}\KeywordTok{str_pad}\NormalTok{(month, }\DecValTok{2}\NormalTok{, }\StringTok{"left"}\NormalTok{, }\StringTok{"0"}\NormalTok{), }\StringTok{".csv"}\NormalTok{))) }
    \CommentTok{#create a df of file path of the files that are in the raw directory}
\NormalTok{    src <-}\StringTok{ }\KeywordTok{list.files}\NormalTok{(}\KeywordTok{attr}\NormalTok{(obj, }\StringTok{"raw_dir"}\NormalTok{), }\StringTok{"yellow"}\NormalTok{, }\DataTypeTok{full.names =} \OtherTok{TRUE}\NormalTok{)}
\NormalTok{    src_small <-}\StringTok{ }\KeywordTok{intersect}\NormalTok{(src, remote}\OperatorTok{$}\NormalTok{src)}
    \CommentTok{#Move the files}
\NormalTok{    in_raw <-}\StringTok{ }\KeywordTok{basename}\NormalTok{(src_small)}
\NormalTok{    in_load <-}\StringTok{ }\KeywordTok{basename}\NormalTok{(}\KeywordTok{list.files}\NormalTok{(}\KeywordTok{attr}\NormalTok{(obj, }\StringTok{"load_dir"}\NormalTok{), }\StringTok{"yellow"}\NormalTok{, }
    \DataTypeTok{full.names =} \OtherTok{TRUE}\NormalTok{))}
\NormalTok{    file_remian <-}\StringTok{ }\KeywordTok{setdiff}\NormalTok{(in_raw,in_load)}
    \KeywordTok{file.copy}\NormalTok{(}\KeywordTok{file.path}\NormalTok{(}\KeywordTok{attr}\NormalTok{(obj, }\StringTok{"raw_dir"}\NormalTok{),file_remian),}
              \KeywordTok{file.path}\NormalTok{(}\KeywordTok{attr}\NormalTok{(obj, }\StringTok{"load_dir"}\NormalTok{),file_remian) )\}}
  \CommentTok{#TAXI GREEN-----------------------------}
\NormalTok{  taxi_green <-}\StringTok{ }\ControlFlowTok{function}\NormalTok{(obj, years, months) \{}
    \KeywordTok{message}\NormalTok{(}\StringTok{"Transforming green taxi data from raw }
\StringTok{            to load directory..."}\NormalTok{)}
    \CommentTok{#create a df of file path of the files that the user wants to transform}
\NormalTok{    remote <-}\StringTok{ }\NormalTok{etl}\OperatorTok{::}\KeywordTok{valid_year_month}\NormalTok{(years, months, }
    \DataTypeTok{begin =} \StringTok{"2013-08-01"}\NormalTok{) }\OperatorTok
\StringTok{      }\KeywordTok{mutate_}\NormalTok{(}\DataTypeTok{src =} \OperatorTok{~}\KeywordTok{file.path}\NormalTok{(}\KeywordTok{attr}\NormalTok{(obj, }\StringTok{"raw_dir"}\NormalTok{), }
      \KeywordTok{paste0}\NormalTok{(}\StringTok{"green"}\NormalTok{, }\StringTok{"_tripdata_"}\NormalTok{, year, }\StringTok{"-"}\NormalTok{,}
\NormalTok{      stringr}\OperatorTok{::}\KeywordTok{str_pad}\NormalTok{(month, }\DecValTok{2}\NormalTok{, }\StringTok{"left"}\NormalTok{, }\StringTok{"0"}\NormalTok{), }\StringTok{".csv"}\NormalTok{))) }
    \CommentTok{#create a df of file path of the files that are in the raw directory}
\NormalTok{    src <-}\StringTok{ }\KeywordTok{list.files}\NormalTok{(}\KeywordTok{attr}\NormalTok{(obj, }\StringTok{"raw_dir"}\NormalTok{), }\StringTok{"green"}\NormalTok{, }\DataTypeTok{full.names =} \OtherTok{TRUE}\NormalTok{)}
\NormalTok{    src_small <-}\StringTok{ }\KeywordTok{intersect}\NormalTok{(src, remote}\OperatorTok{$}\NormalTok{src)}
    \CommentTok{#Clean the green taxi data files}
    \CommentTok{#get rid of 2nd blank row}
    \ControlFlowTok{if}\NormalTok{ (}\KeywordTok{length}\NormalTok{(src_small) }\OperatorTok{==}\StringTok{ }\DecValTok{0}\NormalTok{)\{}
      \KeywordTok{message}\NormalTok{(}\StringTok{"The files you requested are not available }
\StringTok{              in the raw directory."}\NormalTok{)}
\NormalTok{    \} }\ControlFlowTok{else}\NormalTok{\{}
      \CommentTok{#a list of the ones that have a 2nd blank row}
\NormalTok{      remote_green_}\DecValTok{1}\NormalTok{ <-}\StringTok{ }\NormalTok{remote }\OperatorTok\StringTok{ }\KeywordTok{filter_}\NormalTok{(}\OperatorTok{~}\NormalTok{year }\OperatorTok{!=}\StringTok{ }\DecValTok{2015}\NormalTok{)}
\NormalTok{      src_small_green_}\DecValTok{1}\NormalTok{ <-}\StringTok{ }\KeywordTok{intersect}\NormalTok{(src, remote_green_}\DecValTok{1}\OperatorTok{$}\NormalTok{src)}
      \CommentTok{# check that the sys support command line, }
      \CommentTok{#and then remove the blank 2nd row}
      \ControlFlowTok{if}\NormalTok{(}\KeywordTok{length}\NormalTok{(src_small_green_}\DecValTok{1}\NormalTok{) }\OperatorTok{!=}\StringTok{ }\DecValTok{0}\NormalTok{) \{}
        \ControlFlowTok{if}\NormalTok{ (.Platform}\OperatorTok{$}\NormalTok{OS.type }\OperatorTok{==}\StringTok{ "unix"}\NormalTok{)\{}
\NormalTok{          cmds_}\DecValTok{1}\NormalTok{ <-}\StringTok{ }\KeywordTok{paste}\NormalTok{(}\StringTok{"sed -i -e '2d'"}\NormalTok{, src_small_green_}\DecValTok{1}\NormalTok{)}
          \KeywordTok{lapply}\NormalTok{(cmds_}\DecValTok{1}\NormalTok{, system)}
\NormalTok{        \} }\ControlFlowTok{else}\NormalTok{ \{}
          \KeywordTok{message}\NormalTok{(}\StringTok{"Windows system does not }
\StringTok{          currently support removing the 2nd blank row}
\StringTok{          in the green taxi datasets. This might affect }
\StringTok{          loading data into SQL..."}\NormalTok{)\}}
\NormalTok{        \}}\ControlFlowTok{else}\NormalTok{ \{}
          \StringTok{"You did not request for any }
\StringTok{          green taxi data, or all the green}
\StringTok{          taxi data you requested are cleaned."}\NormalTok{\}}
      \CommentTok{#fix column number}
\NormalTok{      remote_green_}\DecValTok{2}\NormalTok{ <-}\StringTok{ }\NormalTok{remote }\OperatorTok
\StringTok{        }\KeywordTok{filter_}\NormalTok{(}\OperatorTok{~}\NormalTok{year }\OperatorTok\StringTok{ }\KeywordTok{c}\NormalTok{(}\DecValTok{2013}\NormalTok{, }\DecValTok{2014}\NormalTok{, }\DecValTok{2015}\NormalTok{)) }\OperatorTok
\StringTok{        }\KeywordTok{mutate_}\NormalTok{(}\DataTypeTok{keep =} 
                  \OperatorTok{~}\KeywordTok{ifelse}\NormalTok{(year }\OperatorTok\StringTok{ }\KeywordTok{c}\NormalTok{(}\DecValTok{2013}\NormalTok{,}\DecValTok{2014}\NormalTok{), }\DecValTok{20}\NormalTok{,}\DecValTok{21}\NormalTok{),}
                \DataTypeTok{new_file =} 
                  \OperatorTok{~}\KeywordTok{paste0}\NormalTok{(}\StringTok{"green_tripdata_"}\NormalTok{, year, }\StringTok{"_"}\NormalTok{, }
\NormalTok{                      stringr}\OperatorTok{::}\KeywordTok{str_pad}\NormalTok{(month, }\DecValTok{2}\NormalTok{, }\StringTok{"left"}\NormalTok{, }\StringTok{"0"}\NormalTok{),}
                                   \StringTok{".csv"}\NormalTok{))}
\NormalTok{      src_small_green_}\DecValTok{2}\NormalTok{ <-}\StringTok{ }\KeywordTok{intersect}\NormalTok{(src, remote_green_}\DecValTok{2}\OperatorTok{$}\NormalTok{src)}
\NormalTok{      src_small_green_2_df <-}\StringTok{ }\KeywordTok{data.frame}\NormalTok{(src_small_green_}\DecValTok{2}\NormalTok{) }
      \KeywordTok{names}\NormalTok{(src_small_green_2_df) <-}\StringTok{ "src"}
\NormalTok{      src_small_green_2_df <-}\StringTok{ }\KeywordTok{inner_join}\NormalTok{(src_small_green_2_df, }
\NormalTok{      remote_green_}\DecValTok{2}\NormalTok{, }\DataTypeTok{by =} \StringTok{"src"}\NormalTok{)}
\NormalTok{      src_small_green_2_df <-}\StringTok{ }\NormalTok{src_small_green_2_df }\OperatorTok
\StringTok{        }\KeywordTok{mutate}\NormalTok{(}\DataTypeTok{cmds_2 =} \KeywordTok{paste}\NormalTok{(}\StringTok{"cut -d, -f1-"}\NormalTok{, keep,}\StringTok{" "}\NormalTok{,src, }\StringTok{" > "}\NormalTok{,}
        \KeywordTok{attr}\NormalTok{(obj, }\StringTok{"raw_dir"}\NormalTok{),}\StringTok{"/green_tripdata_"}\NormalTok{, }
\NormalTok{        year, }\StringTok{"_"}\NormalTok{, stringr}\OperatorTok{::}\KeywordTok{str_pad}\NormalTok{(month, }\DecValTok{2}\NormalTok{, }\StringTok{"left"}\NormalTok{, }\StringTok{"0"}\NormalTok{),}\StringTok{".csv"}\NormalTok{, }
        \DataTypeTok{sep =} \StringTok{""}\NormalTok{))}
      \CommentTok{#remove the extra column}
      \ControlFlowTok{if}\NormalTok{(}\KeywordTok{length}\NormalTok{(src_small_green_}\DecValTok{2}\NormalTok{) }\OperatorTok{!=}\StringTok{ }\DecValTok{0}\NormalTok{) \{}
        \ControlFlowTok{if}\NormalTok{ (.Platform}\OperatorTok{$}\NormalTok{OS.type }\OperatorTok{==}\StringTok{ "unix"}\NormalTok{)\{}
          \KeywordTok{lapply}\NormalTok{(src_small_green_2_df}\OperatorTok{$}\NormalTok{cmds_}\DecValTok{2}\NormalTok{, system)\} }
        \ControlFlowTok{else}\NormalTok{ \{}
          \KeywordTok{message}\NormalTok{(}\StringTok{"Windows system does not currently }
\StringTok{          support removing the 2nd blank row }
\StringTok{          in the green taxi datasets. This might }
\StringTok{          affect loading data into SQL..."}\NormalTok{)\}}
\NormalTok{        \}}\ControlFlowTok{else}\NormalTok{ \{}
          \StringTok{"All the green taxi data you}
\StringTok{          requested are in cleaned formats."}\NormalTok{\}}
      \CommentTok{#Find the files paths of the files that need to be transformed}
      \KeywordTok{file.rename}\NormalTok{(}\KeywordTok{file.path}\NormalTok{(}\KeywordTok{dirname}\NormalTok{(src_small_green_2_df}\OperatorTok{$}\NormalTok{src),}
\NormalTok{                            src_small_green_2_df}\OperatorTok{$}\NormalTok{new_file), }
                  \KeywordTok{file.path}\NormalTok{(}\KeywordTok{attr}\NormalTok{(obj, }\StringTok{"load_dir"}\NormalTok{),}
                  \KeywordTok{basename}\NormalTok{(src_small_green_2_df}\OperatorTok{$}\NormalTok{src)))}
      \CommentTok{#Move the files}
\NormalTok{      in_raw <-}\StringTok{ }\KeywordTok{basename}\NormalTok{(src_small)}
\NormalTok{      in_load <-}\StringTok{ }\KeywordTok{basename}\NormalTok{(}\KeywordTok{list.files}\NormalTok{(}\KeywordTok{attr}\NormalTok{(obj, }\StringTok{"load_dir"}\NormalTok{), }
      \StringTok{"green"}\NormalTok{, }\DataTypeTok{full.names =} \OtherTok{TRUE}\NormalTok{))}
\NormalTok{      file_remian <-}\StringTok{ }\KeywordTok{setdiff}\NormalTok{(in_raw,in_load)}
      \KeywordTok{file.copy}\NormalTok{(}\KeywordTok{file.path}\NormalTok{(}\KeywordTok{attr}\NormalTok{(obj, }\StringTok{"raw_dir"}\NormalTok{),file_remian), }
      \KeywordTok{file.path}\NormalTok{(}\KeywordTok{attr}\NormalTok{(obj, }\StringTok{"load_dir"}\NormalTok{),file_remian) )\}\}}
  \CommentTok{#UBER--------------------------------}
\NormalTok{  uber <-}\StringTok{ }\ControlFlowTok{function}\NormalTok{(obj) \{}
    \KeywordTok{message}\NormalTok{(}\StringTok{"Transforming uber data from raw to load directory..."}\NormalTok{)}
    \CommentTok{#creat a list of 2014 uber data file directory}
\NormalTok{    uber14_list <-}\StringTok{ }\KeywordTok{list.files}\NormalTok{(}\DataTypeTok{path =} \KeywordTok{attr}\NormalTok{(obj, }\StringTok{"raw_dir"}\NormalTok{), }
    \DataTypeTok{pattern =} \StringTok{"14.csv"}\NormalTok{)}
\NormalTok{    uber14_list <-}\StringTok{ }\KeywordTok{data.frame}\NormalTok{(uber14_list)}
\NormalTok{    uber14_list <-}\StringTok{ }\NormalTok{uber14_list }\OperatorTok\StringTok{ }\KeywordTok{mutate_}\NormalTok{(}\DataTypeTok{file_path =} 
    \OperatorTok{~}\KeywordTok{file.path}\NormalTok{(}\KeywordTok{attr}\NormalTok{(obj, }\StringTok{"raw_dir"}\NormalTok{), uber14_list))}
\NormalTok{    uber14file <-}\StringTok{ }\KeywordTok{lapply}\NormalTok{(uber14_list}\OperatorTok{$}\NormalTok{file_path, readr}\OperatorTok{::}\NormalTok{read_csv)}
\NormalTok{    n <-}\StringTok{ }\KeywordTok{length}\NormalTok{(uber14file)}
    \ControlFlowTok{if}\NormalTok{ (n }\OperatorTok{==}\StringTok{ }\DecValTok{1}\NormalTok{) \{}
\NormalTok{      uber14 <-}\StringTok{ }\KeywordTok{data.frame}\NormalTok{(uber14file[}\DecValTok{1}\NormalTok{])}
\NormalTok{    \} }\ControlFlowTok{else} \ControlFlowTok{if}\NormalTok{ (n }\OperatorTok{==}\StringTok{ }\DecValTok{2}\NormalTok{) \{}
\NormalTok{      uber14 <-}\StringTok{ }\KeywordTok{bind_rows}\NormalTok{(uber14file[}\DecValTok{1}\NormalTok{], uber14file[}\DecValTok{2}\NormalTok{])}
\NormalTok{    \} }\ControlFlowTok{else} \ControlFlowTok{if}\NormalTok{ (n }\OperatorTok{>}\StringTok{ }\DecValTok{2}\NormalTok{) \{}
\NormalTok{      uber14 <-}\StringTok{ }\KeywordTok{bind_rows}\NormalTok{(uber14file[}\DecValTok{1}\NormalTok{], uber14file[}\DecValTok{2}\NormalTok{])}
      \ControlFlowTok{for}\NormalTok{ (i }\ControlFlowTok{in} \DecValTok{3}\OperatorTok{:}\NormalTok{n)\{uber14 <-}\StringTok{ }\KeywordTok{bind_rows}\NormalTok{(uber14, uber14file[i])\}}
\NormalTok{    \}}
\NormalTok{    substrRight <-}\StringTok{ }\ControlFlowTok{function}\NormalTok{(x, n)\{}\KeywordTok{substr}\NormalTok{(x, }\KeywordTok{nchar}\NormalTok{(x)}\OperatorTok{-}\NormalTok{n}\OperatorTok{+}\DecValTok{1}\NormalTok{, }\KeywordTok{nchar}\NormalTok{(x))\}}
\NormalTok{    uber14_datetime <-}\StringTok{ }\NormalTok{uber14 }\OperatorTok
\StringTok{      }\KeywordTok{mutate}\NormalTok{(}\DataTypeTok{date =} \KeywordTok{gsub}\NormalTok{( }\StringTok{" .*$"}\NormalTok{, }\StringTok{""}\NormalTok{, }\StringTok{`}\DataTypeTok{Date/Time}\StringTok{`}\NormalTok{), }
      \DataTypeTok{len_date =} \KeywordTok{nchar}\NormalTok{(date), }
             \DataTypeTok{time =} \KeywordTok{sub}\NormalTok{(}\StringTok{'.*}\CharTok{\textbackslash{}\textbackslash{}}\StringTok{ '}\NormalTok{, }\StringTok{''}\NormalTok{, }\StringTok{`}\DataTypeTok{Date/Time}\StringTok{`}\NormalTok{))}
\NormalTok{    uber14_datetime <-}\StringTok{ }\NormalTok{uber14_datetime }\OperatorTok
\StringTok{      }\KeywordTok{mutate}\NormalTok{(}\DataTypeTok{month =} 
               \KeywordTok{substr}\NormalTok{(}\StringTok{`}\DataTypeTok{Date/Time}\StringTok{`}\NormalTok{, }\DecValTok{1}\NormalTok{, }\DecValTok{1}\NormalTok{),}
             \DataTypeTok{day =} \KeywordTok{ifelse}\NormalTok{(len_date }\OperatorTok{==}\StringTok{ }\DecValTok{8}\NormalTok{, }
             \KeywordTok{substr}\NormalTok{(}\StringTok{`}\DataTypeTok{Date/Time}\StringTok{`}\NormalTok{, }\DecValTok{3}\NormalTok{,}\DecValTok{3}\NormalTok{),}\KeywordTok{substr}\NormalTok{(}\StringTok{`}\DataTypeTok{Date/Time}\StringTok{`}\NormalTok{, }\DecValTok{3}\NormalTok{,}\DecValTok{4}\NormalTok{)),}
             \DataTypeTok{pickup_date =} 
\NormalTok{               lubridate}\OperatorTok{::}\KeywordTok{ymd_hms}\NormalTok{(}\KeywordTok{paste0}\NormalTok{(}\StringTok{"2014-"}\NormalTok{, month, }\StringTok{"-"}\NormalTok{, }
\NormalTok{                                         day, }\StringTok{" "}\NormalTok{, time)))}
\NormalTok{    uber14_df <-}\StringTok{ }\NormalTok{uber14_datetime[}\OperatorTok{-}\KeywordTok{c}\NormalTok{(}\DecValTok{1}\NormalTok{,}\DecValTok{5}\OperatorTok{:}\DecValTok{9}\NormalTok{)]}
    
    \CommentTok{#2015}
\NormalTok{    zipped_uberfileURL <-}\StringTok{ }\KeywordTok{file.path}\NormalTok{(}\KeywordTok{attr}\NormalTok{(obj, }\StringTok{"raw_dir"}\NormalTok{),}
    \StringTok{"uber-raw-data-janjune-15.csv.zip"}\NormalTok{)}
\NormalTok{    raw_month_}\DecValTok{2015}\NormalTok{ <-}\StringTok{ }\NormalTok{etl}\OperatorTok{::}\KeywordTok{valid_year_month}\NormalTok{(}\DataTypeTok{years =} \DecValTok{2015}\NormalTok{, }\DataTypeTok{months =} \DecValTok{1}\OperatorTok{:}\DecValTok{6}\NormalTok{)}
\NormalTok{    remote_}\DecValTok{2015}\NormalTok{ <-}\StringTok{ }\NormalTok{etl}\OperatorTok{::}\KeywordTok{valid_year_month}\NormalTok{(years, months)}
\NormalTok{    remote_small_}\DecValTok{2015}\NormalTok{ <-}\StringTok{ }\KeywordTok{inner_join}\NormalTok{(raw_month_}\DecValTok{2015}\NormalTok{, remote_}\DecValTok{2015}\NormalTok{)}
    \ControlFlowTok{if}\NormalTok{(}\KeywordTok{file.exists}\NormalTok{(zipped_uberfileURL) }\OperatorTok{&&}\StringTok{ }
\StringTok{       }\KeywordTok{nrow}\NormalTok{(remote_small_}\DecValTok{2015}\NormalTok{) }\OperatorTok{!=}\StringTok{ }\DecValTok{0}\NormalTok{)\{}
\NormalTok{      utils}\OperatorTok{::}\KeywordTok{unzip}\NormalTok{(}\DataTypeTok{zipfile =}\NormalTok{ zipped_uberfileURL,}\DataTypeTok{unzip =} \StringTok{"internal"}\NormalTok{,}
      \DataTypeTok{exdir =} \KeywordTok{file.path}\NormalTok{(}\KeywordTok{tempdir}\NormalTok{(), }\StringTok{"uber-raw-data-janjune-15.csv.zip"}\NormalTok{))}
\NormalTok{      uber15 <-}\StringTok{ }\NormalTok{readr}\OperatorTok{::}\KeywordTok{read_csv}\NormalTok{(}\KeywordTok{file.path}\NormalTok{(}\KeywordTok{tempdir}\NormalTok{(),}
      \StringTok{"uber-raw-data-janjune-15.csv.zip"}\NormalTok{,}
      \StringTok{"uber-raw-data-janjune-15.csv"}\NormalTok{))\}}
    
    \KeywordTok{names}\NormalTok{(uber14_df) <-}\StringTok{ }\KeywordTok{c}\NormalTok{(}\StringTok{"lat"}\NormalTok{, }\StringTok{"lon"}\NormalTok{, }\StringTok{"affiliated_base_num"}\NormalTok{, }
    \StringTok{"pickup_date"}\NormalTok{)}
    \KeywordTok{names}\NormalTok{(uber15) <-}\StringTok{ }\KeywordTok{tolower}\NormalTok{(}\KeywordTok{names}\NormalTok{(uber15))}
\NormalTok{    uber <-}\StringTok{ }\KeywordTok{bind_rows}\NormalTok{(uber14_df, uber15)}
\NormalTok{    utils}\OperatorTok{::}\KeywordTok{write.csv}\NormalTok{(uber, }\KeywordTok{file.path}\NormalTok{(}\KeywordTok{tempdir}\NormalTok{() ,}\StringTok{"uber.csv"}\NormalTok{))}
    \ControlFlowTok{if}\NormalTok{(}\KeywordTok{nrow}\NormalTok{(uber) }\OperatorTok{!=}\StringTok{ }\DecValTok{0}\NormalTok{) \{}
      \ControlFlowTok{if}\NormalTok{ (.Platform}\OperatorTok{$}\NormalTok{OS.type }\OperatorTok{==}\StringTok{ "unix"}\NormalTok{)\{cmds_}\DecValTok{3}\NormalTok{ <-}\StringTok{ }
\StringTok{      }\KeywordTok{paste}\NormalTok{(}\StringTok{"cut -d, -f2-7"}\NormalTok{,}\KeywordTok{file.path}\NormalTok{(}\KeywordTok{tempdir}\NormalTok{(),}\StringTok{"uber.csv"}\NormalTok{), }\StringTok{" > "}\NormalTok{, }
      \KeywordTok{file.path}\NormalTok{(}\KeywordTok{attr}\NormalTok{(obj, }\StringTok{"load_dir"}\NormalTok{),}\StringTok{"uber.csv"}\NormalTok{))}
        \KeywordTok{lapply}\NormalTok{(cmds_}\DecValTok{3}\NormalTok{, system)}
\NormalTok{      \} }\ControlFlowTok{else}\NormalTok{ \{}
        \KeywordTok{message}\NormalTok{(}\StringTok{"Windows system does not currently }
\StringTok{        support removing the 2nd blank row }
\StringTok{        in the green taxi datasets. This might}
\StringTok{        affect loading data into SQL..."}\NormalTok{)\}}
\NormalTok{      \}}\ControlFlowTok{else}\NormalTok{ \{}
        \StringTok{"You did not request for any }
\StringTok{        green taxi data, or all the green }
\StringTok{        taxi data you requested are cleaned."}\NormalTok{\}}
\NormalTok{    \}}
  \CommentTok{#LYFT--------------------------------}
\NormalTok{  lyft <-}\StringTok{ }\ControlFlowTok{function}\NormalTok{(obj, years, months)\{}
\NormalTok{    valid_months <-}\StringTok{ }\NormalTok{etl}\OperatorTok{::}\KeywordTok{valid_year_month}\NormalTok{(years, }\DataTypeTok{months =} \DecValTok{1}\NormalTok{, }
    \DataTypeTok{begin =} \StringTok{"2015-01-01"}\NormalTok{)}
    \KeywordTok{message}\NormalTok{(}\StringTok{"Transforming lyft data from raw to load directory..."}\NormalTok{)}
\NormalTok{    src <-}\StringTok{ }\KeywordTok{list.files}\NormalTok{(}\KeywordTok{attr}\NormalTok{(obj, }\StringTok{"raw_dir"}\NormalTok{), }\StringTok{"lyft"}\NormalTok{, }\DataTypeTok{full.names =} \OtherTok{TRUE}\NormalTok{)}
\NormalTok{    src_year <-}\StringTok{ }\NormalTok{valid_months }\OperatorTok\StringTok{ }\KeywordTok{distinct_}\NormalTok{(}\OperatorTok{~}\NormalTok{year)}
\NormalTok{    remote <-}\StringTok{ }\KeywordTok{data_frame}\NormalTok{(src)}
\NormalTok{    remote <-}\StringTok{ }\NormalTok{remote }\OperatorTok
\StringTok{      }\KeywordTok{mutate_}\NormalTok{(}\DataTypeTok{lcl =} \OperatorTok{~}\KeywordTok{file.path}\NormalTok{(}\KeywordTok{attr}\NormalTok{(obj, }\StringTok{"load_dir"}\NormalTok{),}\KeywordTok{basename}\NormalTok{(src)),}
              \DataTypeTok{basename =} \OperatorTok{~}\KeywordTok{basename}\NormalTok{(src), }\DataTypeTok{year =} \OperatorTok{~}\KeywordTok{substr}\NormalTok{(basename,}\DecValTok{6}\NormalTok{,}\DecValTok{9}\NormalTok{))}
    \KeywordTok{class}\NormalTok{(remote}\OperatorTok{$}\NormalTok{year) <-}\StringTok{ "numeric"}
\NormalTok{    remote <-}\StringTok{ }\KeywordTok{inner_join}\NormalTok{(remote,src_year, }\DataTypeTok{by =} \StringTok{"year"}\NormalTok{ )}
    \ControlFlowTok{for}\NormalTok{(i }\ControlFlowTok{in} \DecValTok{1}\OperatorTok{:}\KeywordTok{nrow}\NormalTok{(remote)) \{}
\NormalTok{        datafile <-}\StringTok{ }\NormalTok{readr}\OperatorTok{::}\KeywordTok{read_csv}\NormalTok{(remote}\OperatorTok{$}\NormalTok{src[i])}
\NormalTok{        readr}\OperatorTok{::}\KeywordTok{write_delim}\NormalTok{(datafile, }\DataTypeTok{path =}\NormalTok{ remote}\OperatorTok{$}\NormalTok{lcl[i], }
        \DataTypeTok{delim =} \StringTok{"|"}\NormalTok{, }\DataTypeTok{na =} \StringTok{""}\NormalTok{)\}\}}
  
  \CommentTok{#transform the data from raw to load}
  \ControlFlowTok{if}\NormalTok{ (type }\OperatorTok{==}\StringTok{ "yellow"}\NormalTok{)\{}\KeywordTok{taxi_yellow}\NormalTok{(obj, years, months)\} }
  \ControlFlowTok{else} \ControlFlowTok{if}\NormalTok{ (type }\OperatorTok{==}\StringTok{ "green"}\NormalTok{)\{}\KeywordTok{taxi_green}\NormalTok{(obj, years, months)\}}
  \ControlFlowTok{else} \ControlFlowTok{if}\NormalTok{ (type }\OperatorTok{==}\StringTok{ "uber"}\NormalTok{)\{}\KeywordTok{uber}\NormalTok{(obj)\}}
  \ControlFlowTok{else} \ControlFlowTok{if}\NormalTok{ (type }\OperatorTok{==}\StringTok{ "lyft"}\NormalTok{)\{}\KeywordTok{lyft}\NormalTok{(obj, years, months)\}}
  \ControlFlowTok{else}\NormalTok{ \{}\KeywordTok{message}\NormalTok{(}\StringTok{"The type you chose does not exit..."}\NormalTok{)\}}
  
  \KeywordTok{invisible}\NormalTok{(obj)}
\NormalTok{\}}
\end{Highlighting}
\end{Shaded}
\subsection{ETL Load}\label{etl-load}
\begin{Shaded}
\begin{Highlighting}[]
\NormalTok{opts_chunk}\OperatorTok{$}\KeywordTok{set}\NormalTok{(}\DataTypeTok{tidy.opts=}\KeywordTok{list}\NormalTok{(}\DataTypeTok{width.cutoff=}\DecValTok{60}\NormalTok{))}
\NormalTok{etl_load.etl_nyctaxi <-}\StringTok{ }\ControlFlowTok{function}\NormalTok{(obj, }
 \DataTypeTok{years =} \KeywordTok{as.numeric}\NormalTok{(}\KeywordTok{format}\NormalTok{(}\KeywordTok{Sys.Date}\NormalTok{(),}\StringTok{'%Y'}\NormalTok{)), }
                                 \DataTypeTok{months =} \DecValTok{1}\OperatorTok{:}\DecValTok{12}\NormalTok{, }
                                 \DataTypeTok{type  =} \StringTok{"yellow"}\NormalTok{, ...) \{}
  \CommentTok{#TAXI YELLOW-----------------------------}
\NormalTok{  taxi_yellow <-}\StringTok{ }\ControlFlowTok{function}\NormalTok{(obj, years, months,...) \{}
    \CommentTok{#create a df of file path of the files that are in the load directory}
\NormalTok{    src <-}\StringTok{ }\KeywordTok{list.files}\NormalTok{(}\KeywordTok{attr}\NormalTok{(obj, }\StringTok{"load_dir"}\NormalTok{), }\StringTok{"yellow"}\NormalTok{, }
    \DataTypeTok{full.names =} \OtherTok{TRUE}\NormalTok{)}
\NormalTok{    src <-}\StringTok{ }\KeywordTok{data.frame}\NormalTok{(src)}
    
    \CommentTok{#files before 2016-07}
\NormalTok{    remote_old <-}\StringTok{ }\NormalTok{etl}\OperatorTok{::}\KeywordTok{valid_year_month}\NormalTok{(years, months, }
    \DataTypeTok{begin =} \StringTok{"2009-01-01"}\NormalTok{, }\DataTypeTok{end =} \StringTok{"2016-06-30"}\NormalTok{) }\OperatorTok
\StringTok{      }\KeywordTok{mutate_}\NormalTok{(}\DataTypeTok{src =} \OperatorTok{~}\KeywordTok{file.path}\NormalTok{(}\KeywordTok{attr}\NormalTok{(obj, }\StringTok{"load_dir"}\NormalTok{), }
      \KeywordTok{paste0}\NormalTok{(}\StringTok{"yellow"}\NormalTok{, }\StringTok{"_tripdata_"}\NormalTok{, year, }\StringTok{"-"}\NormalTok{,}
\NormalTok{      stringr}\OperatorTok{::}\KeywordTok{str_pad}\NormalTok{(month, }\DecValTok{2}\NormalTok{, }\StringTok{"left"}\NormalTok{, }\StringTok{"0"}\NormalTok{), }\StringTok{".csv"}\NormalTok{))) }
\NormalTok{    src_small_old <-}\StringTok{ }\KeywordTok{inner_join}\NormalTok{(remote_old, src, }\DataTypeTok{by =} \StringTok{"src"}\NormalTok{)}
    \CommentTok{#files later then 2017-06}
\NormalTok{    remote_new <-}\StringTok{ }\NormalTok{etl}\OperatorTok{::}\KeywordTok{valid_year_month}\NormalTok{(years, months, }
    \DataTypeTok{begin =} \StringTok{"2016-07-01"}\NormalTok{) }\OperatorTok
\StringTok{      }\KeywordTok{mutate_}\NormalTok{(}\DataTypeTok{src =}  \OperatorTok{~}\KeywordTok{file.path}\NormalTok{(}\KeywordTok{attr}\NormalTok{(obj, }\StringTok{"load_dir"}\NormalTok{), }
      \KeywordTok{paste0}\NormalTok{(}\StringTok{"yellow"}\NormalTok{, }\StringTok{"_tripdata_"}\NormalTok{, year, }\StringTok{"-"}\NormalTok{,}
\NormalTok{      stringr}\OperatorTok{::}\KeywordTok{str_pad}\NormalTok{(month, }\DecValTok{2}\NormalTok{, }\StringTok{"left"}\NormalTok{, }\StringTok{"0"}\NormalTok{), }\StringTok{".csv"}\NormalTok{))) }
\NormalTok{    src_small_new <-}\StringTok{ }\KeywordTok{inner_join}\NormalTok{(remote_new, src, }\DataTypeTok{by =} \StringTok{"src"}\NormalTok{)}
    \CommentTok{#data earlier than 2016-07}
    \ControlFlowTok{if}\NormalTok{(}\KeywordTok{nrow}\NormalTok{(src_small_old) }\OperatorTok{==}\StringTok{ }\DecValTok{0}\NormalTok{) \{}
      \KeywordTok{message}\NormalTok{(}\StringTok{"The taxi files (earlier than 2016-07) }
\StringTok{              you requested are not available in }
\StringTok{              the load directory..."}\NormalTok{)}
\NormalTok{    \} }\ControlFlowTok{else}\NormalTok{ \{}
      \KeywordTok{message}\NormalTok{(}\StringTok{"Loading taxi data from }
\StringTok{              load directory to a sql database..."}\NormalTok{)}
      \KeywordTok{mapply}\NormalTok{(DBI}\OperatorTok{::}\NormalTok{dbWriteTable, }
             \DataTypeTok{name =} \StringTok{"yellow_old"}\NormalTok{, }\DataTypeTok{value =}\NormalTok{ src_small_old}\OperatorTok{$}\NormalTok{src, }
             \DataTypeTok{MoreArgs =} 
               \KeywordTok{list}\NormalTok{(}\DataTypeTok{conn =}\NormalTok{ obj}\OperatorTok{$}\NormalTok{con, }\DataTypeTok{append =} \OtherTok{TRUE}\NormalTok{))\}}
    
    \CommentTok{#data later then 2016-06}
    \ControlFlowTok{if}\NormalTok{(}\KeywordTok{nrow}\NormalTok{(src_small_new) }\OperatorTok{==}\StringTok{ }\DecValTok{0}\NormalTok{) \{}
      \KeywordTok{message}\NormalTok{(}\StringTok{"The new taxi files (later than 2016-06) }
\StringTok{              you requested are not available in the }
\StringTok{              load directory..."}\NormalTok{)}
\NormalTok{    \} }\ControlFlowTok{else}\NormalTok{ \{}
      \KeywordTok{message}\NormalTok{(}\StringTok{"Loading taxi data from load }
\StringTok{              directory to a sql database..."}\NormalTok{)}
      \KeywordTok{mapply}\NormalTok{(DBI}\OperatorTok{::}\NormalTok{dbWriteTable, }
             \DataTypeTok{name =} \StringTok{"yellow"}\NormalTok{, }\DataTypeTok{value =}\NormalTok{ src_small_new}\OperatorTok{$}\NormalTok{src, }
             \DataTypeTok{MoreArgs =} 
               \KeywordTok{list}\NormalTok{(}\DataTypeTok{conn =}\NormalTok{ obj}\OperatorTok{$}\NormalTok{con, }\DataTypeTok{append =} \OtherTok{TRUE}\NormalTok{))\}}
    
\NormalTok{    \}}
  \CommentTok{#TAXI GREEN----------------------------}
\NormalTok{  taxi_green <-}\StringTok{ }\ControlFlowTok{function}\NormalTok{(obj, years, months,...) \{}
    \CommentTok{#create a list of file that the user wants to load}
\NormalTok{    remote <-}\StringTok{ }\NormalTok{etl}\OperatorTok{::}\KeywordTok{valid_year_month}\NormalTok{(years, months, }
    \DataTypeTok{begin =} \StringTok{"2013-08-01"}\NormalTok{) }\OperatorTok
\StringTok{      }\KeywordTok{mutate_}\NormalTok{(}\DataTypeTok{src =} \OperatorTok{~}\KeywordTok{file.path}\NormalTok{(}\KeywordTok{attr}\NormalTok{(obj, }\StringTok{"load_dir"}\NormalTok{), }
      \KeywordTok{paste0}\NormalTok{(}\StringTok{"green"}\NormalTok{, }\StringTok{"_tripdata_"}\NormalTok{, year, }\StringTok{"-"}\NormalTok{,}
\NormalTok{      stringr}\OperatorTok{::}\KeywordTok{str_pad}\NormalTok{(month, }\DecValTok{2}\NormalTok{, }\StringTok{"left"}\NormalTok{, }\StringTok{"0"}\NormalTok{), }\StringTok{".csv"}\NormalTok{)))}
    \CommentTok{#create a df of file path of the files that are in the load directory}
\NormalTok{    src <-}\StringTok{ }\KeywordTok{list.files}\NormalTok{(}\KeywordTok{attr}\NormalTok{(obj, }\StringTok{"load_dir"}\NormalTok{), }\StringTok{"tripdata"}\NormalTok{, }
    \DataTypeTok{full.names =} \OtherTok{TRUE}\NormalTok{)}
\NormalTok{    src <-}\StringTok{ }\KeywordTok{data.frame}\NormalTok{(src)}
    \CommentTok{#only keep the files thst the user wants to transform}
\NormalTok{    src_small <-}\StringTok{ }\KeywordTok{inner_join}\NormalTok{(remote, src, }\DataTypeTok{by =} \StringTok{"src"}\NormalTok{)}
    \ControlFlowTok{if}\NormalTok{(}\KeywordTok{nrow}\NormalTok{(src_small) }\OperatorTok{==}\StringTok{ }\DecValTok{0}\NormalTok{) \{}
      \KeywordTok{message}\NormalTok{(}\StringTok{"The taxi files you requested }
\StringTok{              are not available in the }
\StringTok{              load directory..."}\NormalTok{)}
\NormalTok{    \} }\ControlFlowTok{else}\NormalTok{ \{}
      \KeywordTok{message}\NormalTok{(}\StringTok{"Loading taxi data from }
\StringTok{              load directory to a sql database..."}\NormalTok{)}
      \KeywordTok{mapply}\NormalTok{(DBI}\OperatorTok{::}\NormalTok{dbWriteTable, }
             \DataTypeTok{name =} \StringTok{"green"}\NormalTok{, }\DataTypeTok{value =}\NormalTok{ src_small}\OperatorTok{$}\NormalTok{src, }
             \DataTypeTok{MoreArgs =} 
               \KeywordTok{list}\NormalTok{(}\DataTypeTok{conn =}\NormalTok{ obj}\OperatorTok{$}\NormalTok{con, }\DataTypeTok{append =} \OtherTok{TRUE}\NormalTok{, }\DataTypeTok{... =}\NormalTok{ ...))\}\}}
  \CommentTok{#UBER--------------------------------}
\NormalTok{  uber <-}\StringTok{ }\ControlFlowTok{function}\NormalTok{(obj,...) \{}
\NormalTok{    uberfileURL <-}\StringTok{ }\KeywordTok{file.path}\NormalTok{(}\KeywordTok{attr}\NormalTok{(obj, }\StringTok{"load_dir"}\NormalTok{), }\StringTok{"uber.csv"}\NormalTok{)}
    \ControlFlowTok{if}\NormalTok{(}\KeywordTok{file.exists}\NormalTok{(uberfileURL)) \{}
      \KeywordTok{message}\NormalTok{(}\StringTok{"Loading uber data from }
\StringTok{              load directory to a sql database..."}\NormalTok{)}
\NormalTok{      DBI}\OperatorTok{::}\KeywordTok{dbWriteTable}\NormalTok{(}\DataTypeTok{conn =}\NormalTok{ obj}\OperatorTok{$}\NormalTok{con, }\DataTypeTok{name =} \StringTok{"uber"}\NormalTok{, }
      \DataTypeTok{value =}\NormalTok{ uberfileURL, }\DataTypeTok{append =} \OtherTok{TRUE}\NormalTok{, }\DataTypeTok{... =}\NormalTok{ ...)}
\NormalTok{    \} }\ControlFlowTok{else}\NormalTok{ \{}
      \KeywordTok{message}\NormalTok{(}\StringTok{"There is no uber data }
\StringTok{              in the load directory..."}\NormalTok{)\}\}}
  \CommentTok{#LYFT---------------------------------}
\NormalTok{  lyft <-}\StringTok{ }\ControlFlowTok{function}\NormalTok{(obj, years, months,...)\{}
    \KeywordTok{message}\NormalTok{(}\StringTok{"Loading lyft data from }
\StringTok{            load directory to a sql database..."}\NormalTok{)}
    \CommentTok{#create a list of file that the user wants to load}
\NormalTok{    valid_months <-}\StringTok{ }\NormalTok{etl}\OperatorTok{::}\KeywordTok{valid_year_month}\NormalTok{(years, months, }
    \DataTypeTok{begin =} \StringTok{"2015-01-01"}\NormalTok{)}
\NormalTok{    src <-}\StringTok{ }\KeywordTok{list.files}\NormalTok{(}\KeywordTok{attr}\NormalTok{(obj, }\StringTok{"load_dir"}\NormalTok{), }\StringTok{"lyft"}\NormalTok{, }
    \DataTypeTok{full.names =} \OtherTok{TRUE}\NormalTok{)}
\NormalTok{    src_year <-}\StringTok{ }\NormalTok{valid_months }\OperatorTok\StringTok{ }\KeywordTok{distinct_}\NormalTok{(}\OperatorTok{~}\NormalTok{year)}
\NormalTok{    remote <-}\StringTok{ }\KeywordTok{data_frame}\NormalTok{(src)}
\NormalTok{    remote <-}\StringTok{ }\NormalTok{remote }\OperatorTok\StringTok{ }\KeywordTok{mutate_}\NormalTok{(}\DataTypeTok{tablename =} \OperatorTok{~}\StringTok{"lyft"}\NormalTok{, }
    \DataTypeTok{year =}\OperatorTok{~}\KeywordTok{substr}\NormalTok{(}\KeywordTok{basename}\NormalTok{(src),}\DecValTok{6}\NormalTok{,}\DecValTok{9}\NormalTok{))}
    \KeywordTok{class}\NormalTok{(remote}\OperatorTok{$}\NormalTok{year) <-}\StringTok{ "numeric"}
\NormalTok{    remote <-}\StringTok{ }\KeywordTok{inner_join}\NormalTok{(remote,src_year, }\DataTypeTok{by =} \StringTok{"year"}\NormalTok{ )}
    \ControlFlowTok{if}\NormalTok{(}\KeywordTok{nrow}\NormalTok{(remote) }\OperatorTok{!=}\StringTok{ }\DecValTok{0}\NormalTok{) \{}
\NormalTok{      write_data <-}\StringTok{ }\ControlFlowTok{function}\NormalTok{(...) \{}
        \KeywordTok{lapply}\NormalTok{(remote}\OperatorTok{$}\NormalTok{src, }\DataTypeTok{FUN =}\NormalTok{ DBI}\OperatorTok{::}\NormalTok{dbWriteTable, }
        \DataTypeTok{conn =}\NormalTok{ obj}\OperatorTok{$}\NormalTok{con, }\DataTypeTok{name =} \StringTok{"lyft"}\NormalTok{, }\DataTypeTok{append =} \OtherTok{TRUE}\NormalTok{, }
        \DataTypeTok{sep =} \StringTok{"|"}\NormalTok{, }\DataTypeTok{... =}\NormalTok{ ...)\}}
      \KeywordTok{write_data}\NormalTok{(...)}
\NormalTok{    \} }\ControlFlowTok{else}\NormalTok{ \{}
      \KeywordTok{message}\NormalTok{(}\StringTok{"The lyft files you requested }
\StringTok{              are not available in the }
\StringTok{              load directory..."}\NormalTok{)\}\}}
  
  \ControlFlowTok{if}\NormalTok{ (type }\OperatorTok{==}\StringTok{ "yellow"}\NormalTok{)\{}\KeywordTok{taxi_yellow}\NormalTok{(obj, years, months,...)}
\NormalTok{  \}}\ControlFlowTok{else} \ControlFlowTok{if}\NormalTok{ (type }\OperatorTok{==}\StringTok{ "green"}\NormalTok{)\{}\KeywordTok{taxi_green}\NormalTok{(obj, years, months,...)}
\NormalTok{  \}}\ControlFlowTok{else} \ControlFlowTok{if}\NormalTok{ (type }\OperatorTok{==}\StringTok{ "uber"}\NormalTok{)\{}\KeywordTok{uber}\NormalTok{(obj,...)}
\NormalTok{  \}}\ControlFlowTok{else} \ControlFlowTok{if}\NormalTok{ (type }\OperatorTok{==}\StringTok{ "lyft"}\NormalTok{)\{}\KeywordTok{lyft}\NormalTok{(obj, years, months,...)}
\NormalTok{  \}}\ControlFlowTok{else}\NormalTok{ \{}\KeywordTok{message}\NormalTok{(}\StringTok{"The type you chose does not exit..."}\NormalTok{)}
\NormalTok{            \}}
  
  \KeywordTok{invisible}\NormalTok{(obj)}
\NormalTok{\}}
\end{Highlighting}
\end{Shaded}
\subsection{ETL Init}\label{etl-init}
\begin{verbatim}
DROP TABLE IF EXISTS `yellow_old`;

CREATE TABLE `yellow_old` (
 `VendorID` tinyint DEFAULT NULL,
 `tpep_pickup_datetime` DATETIME NOT NULL,
 `tpep_dropoff_datetime` DATETIME NOT NULL,
 `passenger_count` tinyint DEFAULT NULL,
 `trip_distance` float(10,2) DEFAULT NULL,
 `pickup_longitude` double(7,5) DEFAULT NULL,
 `pickup_latitude` double(7,5) DEFAULT NULL,
 `RatecodeID` tinyint DEFAULT NULL,
 `store_and_fwd_flag` varchar(10) COLLATE latin1_general_ci DEFAULT NULL,
 `dropoff_longitude` double(7,5) DEFAULT NULL,
 `dropoff_latitude` double(7,5) DEFAULT NULL,
 `payment_type` tinyint DEFAULT NULL,
 `fare_amount` decimal(5,3) DEFAULT NULL,
 `extra` decimal(5,3) DEFAULT NULL,
 `mta_tax` decimal(5,3) DEFAULT NULL,
 `tip_amount` decimal(5,3) DEFAULT NULL,
 `tolls_amount` decimal(5,3) DEFAULT NULL,
 `improvement_surcharge` decimal(5,3) DEFAULT NULL,
 `total_amount` decimal(5,3) DEFAULT NULL,
 KEY `VendorID` (`VendorID`),
 KEY `pickup_datetime` (`tpep_pickup_datetime`),
 KEY `dropoff_datetime` (`tpep_dropoff_datetime`),
 KEY `pickup_longitude` (`pickup_longitude`),
 KEY `pickup_latitude` (`pickup_latitude`),
 KEY `dropoff_longitude` (`dropoff_longitude`),
 KEY `dropoff_latitude` (`dropoff_latitude`)
)
PARTITION BY RANGE( YEAR(tpep_pickup_datetime) ) (
  PARTITION p09 VALUES LESS THAN (2010),
  PARTITION p10 VALUES LESS THAN (2011),
  PARTITION p11 VALUES LESS THAN (2012),
  PARTITION p12 VALUES LESS THAN (2013),
  PARTITION p13 VALUES LESS THAN (2014),
  PARTITION p14 VALUES LESS THAN (2015),
  PARTITION p15 VALUES LESS THAN (2016),
  PARTITION p16 VALUES LESS THAN (2017)
);

DROP TABLE IF EXISTS `yellow`;

CREATE TABLE `yellow` (
 `VendorID` tinyint DEFAULT NULL,
 `tpep_pickup_datetime` DATETIME NOT NULL,
 `tpep_dropoff_datetime` DATETIME NOT NULL,
 `passenger_count` tinyint DEFAULT NULL,
 `trip_distance` float(10,2) DEFAULT NULL,
 `RatecodeID` tinyint DEFAULT NULL,
 `store_and_fwd_flag` varchar(10) COLLATE latin1_general_ci DEFAULT NULL,
 `PULocationID` tinyint DEFAULT NULL,
 `DOLocationID` tinyint DEFAULT NULL,
 `payment_type` tinyint DEFAULT NULL,
 `fare_amount` decimal(5,3) DEFAULT NULL,
 `extra` decimal(5,3) DEFAULT NULL,
 `mta_tax` decimal(5,3) DEFAULT NULL,
 `tip_amount` decimal(5,3) DEFAULT NULL,
 `tolls_amount` decimal(5,3) DEFAULT NULL,
 `improvement_surcharge` decimal(5,3) DEFAULT NULL,
 `total_amount` decimal(5,3) DEFAULT NULL,
 KEY `VendorID` (`VendorID`),
 KEY `pickup_datetime` (`tpep_pickup_datetime`),
 KEY `dropoff_datetime` (`tpep_dropoff_datetime`),
 KEY `PULocationID` (`PULocationID`),
 KEY `DOLocationID` (`DOLocationID`)
)
PARTITION BY RANGE( YEAR(tpep_pickup_datetime) ) (
  PARTITION p16 VALUES LESS THAN (2017),
  PARTITION p17 VALUES LESS THAN (2018)
);


DROP TABLE IF EXISTS `green`;

CREATE TABLE `green` (
 `VendorID` tinyint DEFAULT NULL,
 `lpep_pickup_datetime` DATETIME NOT NULL,
 `Lpep_dropoff_datetime` DATETIME NOT NULL,
 `Store_and_fwd_flag` varchar(10) COLLATE latin1_general_ci DEFAULT NULL,
 `RatecodeID` tinyint DEFAULT NULL,
 `Pickup_longitude` double(7,5) DEFAULT NULL,
 `Pickup_latitude` double(7,5) DEFAULT NULL,
 `Dropoff_longitude` double(7,5) DEFAULT NULL,
 `Dropoff_latitude` double(7,5) DEFAULT NULL,
 `Passenger_count` tinyint DEFAULT NULL,
 `Trip_distance` float(10,2) DEFAULT NULL,
 `Fare_amount` decimal(5,3) DEFAULT NULL,
 `Extra` decimal(5,3) DEFAULT NULL,
 `MTA_tax` decimal(5,3) DEFAULT NULL,
 `Tip_amount` decimal(5,3) DEFAULT NULL,
 `Tolls_amount` decimal(5,3) DEFAULT NULL,
 `improvement_surcharge` decimal(5,3) DEFAULT NULL,
 `Total_amount` decimal(5,3) DEFAULT NULL,
 `Payment_type` tinyint DEFAULT NULL,
 `Trip_type` tinyint DEFAULT NULL,
 KEY `VendorID` (`VendorID`),
 KEY `pickup_datetime` (`lpep_pickup_datetime`),
 KEY `dropoff_datetime` (`Lpep_dropoff_datetime`)
);


DROP TABLE IF EXISTS `lyft`;

CREATE TABLE `lyft` (
 `base_license_number` varchar(15) COLLATE latin1_general_ci DEFAULT NULL,
 `base_name` varchar(40) COLLATE latin1_general_ci DEFAULT NULL,
 `dba` varchar(40) COLLATE latin1_general_ci DEFAULT NULL,
 `pickup_end_date` DATE NOT NULL,
 `pickup_start_date` DATE NOT NULL,
 `total_dispatched_trips` smallint DEFAULT NULL,
 `unique_dispatched_vehicle` smallint DEFAULT NULL,
 `wave_number` tinyint DEFAULT NULL,
 `week_number` tinyint DEFAULT NULL,
 `years` smallint DEFAULT NULL,
 KEY `base_name` (`base_name`),
 KEY `pickup_end_date` (`pickup_end_date`),
 KEY `pickup_start_date` (`pickup_start_date`)
);


DROP TABLE IF EXISTS `uber`;

CREATE TABLE `uber` (
 `lat` double(7,5) DEFAULT NULL,
 `lon` double(7,5) DEFAULT NULL,
 `dispatching_base_num` varchar(15) COLLATE latin1_general_ci DEFAULT NULL,
 `pickup_date` DATETIME NOT NULL,
 `affiliated_base_num` varchar(15) COLLATE latin1_general_ci DEFAULT NULL,
 `locationid` tinyint DEFAULT NULL,
 KEY `pickup_date` (`pickup_date`),
 KEY `locationid` (`locationid`)
);

CREATE VIEW yellow_old_sum AS SELECT YEAR(tpep_pickup_datetime) as the_year, MONTH(tpep_pickup_datetime) AS the_month, count(*) AS num_trips
  FROM yellow_old
  GROUP BY the_year, the_month; 
); 
\end{verbatim}
\chapter{New York City Taxi Driver}\label{chapter3}

The income of Taxi drivers in New York City has two parts: taxi fare and
tips. Taxi fare is usually calculated by the meters installed in the
taxis, and the rate of fare cannot be changed by taxi drivers.
Therefore, in order to make more profit, taxi drivers prefer to pick up
passengers who offer big amount of tips. What are the regions that
provide the most tips to yellow taxicab drivers?

In this analysis, we will focus on trip data collected in 2017.
Descriptions of variables mentioned in the following chapters can be
found in Appendix B.

In order to answer questions regarding to taxi trips' tips, we filter
out trips that are not paid by credit or debit card, because taxi
drivers usually do not correctly record the amount of tips paid by cash
or check (Appendix C) (W. Li, 2018).

As mentioned in the previous chapter, that we can utlize the connection
to a MySQL database to run data analysis in MySQL for medium-sized data.
Since we are using all 12 month data from 2017 in this analysis, it is
impractical to load all data needed into \textbf{R} environment.
Instead, we want to only load a fraction of the 2017 Yellow Taxi data
from MySQL database.

In this section, we only want to load trip records with payment type
equals to 1, which represents credit card. Only trip records with
payment type credit card have accurate information on tip amount. Let's
load the 2017 trip record into \textbf{R} environment by using the MySQL
connection we just generated, \texttt{taxi}.
\begin{Shaded}
\begin{Highlighting}[]
\NormalTok{yellow_}\DecValTok{2017}\NormalTok{ <-}\StringTok{ }\NormalTok{taxi }\OperatorTok
\StringTok{  }\KeywordTok{tbl}\NormalTok{(}\StringTok{"yellow"}\NormalTok{) }\OperatorTok
\StringTok{  }\KeywordTok{filter}\NormalTok{(payment_type }\OperatorTok{==}\StringTok{ }\DecValTok{1}\NormalTok{) }\OperatorTok
\StringTok{  }\KeywordTok{collect}\NormalTok{(}\DataTypeTok{n =} \OtherTok{Inf}\NormalTok{)}
\end{Highlighting}
\end{Shaded}
\section{Aggregated Zone-level Tip
Amount}\label{aggregated-zone-level-tip-amount}

Instead of the nominal amount of tips, we want to focus on the
percentage of tips that passengers pay in addition to the total fare
amount. Therefore, we use tip amount over fare amount to calculate the
percent tip. We then calculated the mean percent tip, mean distances
travelled, mean number of minutes spent travelling, and total number of
trips of each pick-up and drop-off pair in 2017 to get the aggregated
zone-level information in order to compare the percent tip passengers
pay in each zone.
\begin{Shaded}
\begin{Highlighting}[]
\NormalTok{yellow_2017_summary <-}\StringTok{ }\NormalTok{yellow_}\DecValTok{2017} \OperatorTok
\StringTok{  }\KeywordTok{mutate}\NormalTok{(}\DataTypeTok{year =} \KeywordTok{year}\NormalTok{(tpep_pickup_datetime),}
         \DataTypeTok{month =} \KeywordTok{month}\NormalTok{(tpep_pickup_datetime),}
         \DataTypeTok{tip_perct =}\NormalTok{ tip_amount}\OperatorTok{/}\NormalTok{fare_amount) }\OperatorTok
\StringTok{  }\KeywordTok{group_by}\NormalTok{(year, month, PULocationID, DOLocationID) }\OperatorTok
\StringTok{  }\KeywordTok{summarise}\NormalTok{(}\DataTypeTok{avg_tip =} \KeywordTok{mean}\NormalTok{(tip_perct), }
            \DataTypeTok{trips =} \KeywordTok{n}\NormalTok{(),}
            \DataTypeTok{avg_dis =} \KeywordTok{mean}\NormalTok{(trip_distance),}
            \DataTypeTok{avg_duration =} \KeywordTok{mean}\NormalTok{(duration))}
\end{Highlighting}
\end{Shaded}
Each taxi trip has pick-up and drop-off locations associated with it,
and there are 263 known taxi zones and 2 taxi zones that are labelled as
``Unknown''. We only want to include trips coming from and going to
known taxi zones in this analysis.
\begin{figure}

{\centering \includegraphics[width=5.96in]{figure/region_vis} 

}

\caption{Percent Tip Paid by Passengers on Each Pick-up And Drop-off Pair in NYC}\label{fig:region-vis}
\end{figure}
Figure \ref{fig:region-vis} is a histogram of mean tip percents for all
known pick-up and drop-off zone pairs. The red, green, and yellow dash
lines are drawn at 20\%, 25\%, and 30\%, which are the default
percentage of tips that are shown on the touch panel for credit and
debit car payments.
\begin{figure}

{\centering \includegraphics[width=4.8in]{figure/taxi-screen} 

}

\caption{Tip Payment Page on New York City Touch Panel}\label{fig:taxi-screen}
\end{figure}
\subsection{Pick-up Zone Percent Tip
Amount}\label{pick-up-zone-percent-tip-amount}

Taxi drivers are required to be indifferent to where passengers are
going. It is illegal for New York city taxi drivers to refuse service
because of passengers' race, ethnicity, cultural background, disbility,
gender, or destination (Harshbarger, 2015). Taxi drivers cannot choose
where the passengers want to go, and instead they can only choose which
pick-up zone they would prefer to driver around to get hailed.
Therefore, it makes sense to investigate the average amount of tips paid
by passengers departed from each pick-up zone. What are the taxi pick-up
zones that have the highest percent tips paid by passengers?

We first calculate the average percent tip paid for each pick-up zone.
\begin{table}

\caption{\label{tab:unnamed-chunk-40}Ten taxi pick-up zones with the highest average tip in January, 2017}
\centering
\begin{tabular}[t]{ccccc}
\toprule
avg\_tip & avg\_dis & avg\_duration & Borough & Zone\\
\midrule
29 & 7.37 & 15.74 & Queens & Douglaston\\
29 & 7.05 & 20.78 & Bronx & East Tremont\\
29 & 10.40 & 17.87 & Queens & Oakland Gardens\\
28 & 7.97 & 22.29 & Queens & Glendale\\
28 & 8.33 & 25.23 & Queens & Saint Michaels Cemetery/Woodside\\
\addlinespace
27 & 7.60 & 19.70 & Queens & Bayside\\
27 & 9.56 & 24.35 & Brooklyn & Coney Island\\
27 & 10.63 & 25.94 & Queens & Howard Beach\\
27 & 11.20 & 23.28 & Brooklyn & Marine Park/Mill Basin\\
26 & 6.22 & 18.80 & Bronx & Norwood\\
\bottomrule
\end{tabular}
\end{table}
Table 3.1 is a list of pick-up zones with their average percent tips.

We created a histogram to visualize the distribution of average percent
tips paid for all pick-up zones.
\begin{figure}

{\centering \includegraphics[width=4.77in]{figure/pickup_vis} 

}

\caption{Percent Tip Paid by Passengers on Each Pick-up Taxi Zone in NYC}\label{fig:pickup-vis}
\end{figure}
As show in Figure \ref{fig:pickup-vis}, the first peak is around 20\%,
which is the cheapest default option on the touch panel for passengers
to chose.

\subsection{Which pick-up zones have the highest number of
pick-ups?}\label{which-pick-up-zones-have-the-highest-number-of-pick-ups}

Which pick-up zones have the highest number of taxi trip pick-ups? We
can create a heat map to visualizae the number of trips for each pick-up
zones on a map of New York City Taxi Zones.
\begin{figure}

{\centering \includegraphics[width=4.96in]{figure/num_trip} 

}

\caption{Number of Pick-ups in Each Taxi Zone}\label{fig:num-trip}
\end{figure}
According to Figure \ref{fig:num-trip}, it's obvious that Upper East
Side Manhattan, Midtown Manhattan, and La Guardia Airport are the most
popular location for pick-ups.
\begin{table}

\caption{\label{tab:unnamed-chunk-44}Ten taxi zones with the highest number of pick-ups}
\centering
\begin{tabular}[t]{ccccc}
\toprule
avg\_tip & avg\_dis & avg\_duration & Borough & Zone\\
\midrule
19.64 & 9.21 & 33.32 & Manhattan & Upper East Side South\\
20.82 & 10.04 & 34.48 & Manhattan & Midtown Center\\
20.18 & 9.63 & 33.22 & Manhattan & Union Sq\\
19.73 & 9.21 & 31.94 & Manhattan & Upper East Side North\\
20.55 & 9.70 & 32.83 & Manhattan & Midtown East\\
\addlinespace
20.55 & 9.66 & 31.90 & Manhattan & Murray Hill\\
20.55 & 9.99 & 36.77 & Manhattan & Penn Station/Madison Sq West\\
20.00 & 9.48 & 30.39 & Manhattan & East Village\\
21.00 & 11.90 & 36.10 & Queens & LaGuardia Airport\\
20.36 & 10.32 & 35.97 & Manhattan & Times Sq/Theatre District\\
\bottomrule
\end{tabular}
\end{table}
Table 3.2 gives you a better idea of which taxi zones have the highest
number of pick-ups.

\subsection{Which pick-up zones have the highest percent
tips?}\label{which-pick-up-zones-have-the-highest-percent-tips}

Most yellow cab pick-ups occur in Manhattan. If we focus on the pick-up
zones that have at least 24 trips per day or 8760 per year, we will
observe that many taxi pick-up zones with the highest percent tips are
not necessarily the ones with the highest number of pick-ups.
\begin{table}

\caption{\label{tab:unnamed-chunk-46}Ten taxi pick-up zones with the highest percent tip (taxi zones has at least 1 pick-up per hour)}
\centering
\begin{tabular}[t]{ccccc}
\toprule
avg\_tip & avg\_dis & avg\_duration & Borough & Zone\\
\midrule
21.55 & 14.27 & 42.94 & Queens & Baisley Park\\
21.45 & 6.89 & 24.04 & Brooklyn & Gowanus\\
21.36 & 8.18 & 26.10 & Queens & Steinway\\
21.18 & 7.11 & 25.85 & Brooklyn & Carroll Gardens\\
21.00 & 11.90 & 36.10 & Queens & LaGuardia Airport\\
\addlinespace
21.00 & 6.74 & 25.06 & Brooklyn & Greenpoint\\
21.00 & 6.45 & 25.44 & Brooklyn & Prospect Heights\\
20.82 & 10.04 & 34.48 & Manhattan & Midtown Center\\
20.82 & 6.91 & 26.12 & Brooklyn & Cobble Hill\\
20.82 & 6.83 & 23.07 & Brooklyn & East Williamsburg\\
\bottomrule
\end{tabular}
\end{table}
People might think it is more reasonable to see a list that is populated
with Zones in Manhattan, since that's where most of the wealthy people
live. However, Table @ref(tab:pickup\_zone\_30) shows that passengers
who get on taxis from certain zones in Brooklyn and Queens also pay a
lot of tips. Taxi drivers who would love to get more tips compensation
can drive to the zones listed above to pick-up passengers.

If we focus on the pick-up zones that have more than 2400 trips per day,
then we observe that all pick-up zones that have the highest percent
tips are in Manhattan besides La Guardia Airport.
\begin{table}

\caption{\label{tab:unnamed-chunk-48}Ten taxi pick-up zones with the highest percent tip (taxi zones has at least 1 pick-up per minute)}
\centering
\begin{tabular}[t]{ccccc}
\toprule
avg\_tip & avg\_dis & avg\_duration & Borough & Zone\\
\midrule
21.00 & 11.90 & 36.10 & Queens & LaGuardia Airport\\
20.82 & 10.04 & 34.48 & Manhattan & Midtown Center\\
20.73 & 10.22 & 32.82 & Manhattan & Battery Park City\\
20.55 & 9.70 & 32.83 & Manhattan & Midtown East\\
20.55 & 9.66 & 31.90 & Manhattan & Murray Hill\\
\addlinespace
20.55 & 9.99 & 36.77 & Manhattan & Penn Station/Madison Sq West\\
20.45 & 9.07 & 30.18 & Manhattan & UN/Turtle Bay South\\
20.36 & 10.32 & 35.97 & Manhattan & Times Sq/Theatre District\\
20.18 & 9.63 & 33.22 & Manhattan & Union Sq\\
20.18 & 9.84 & 35.49 & Manhattan & Midtown North\\
\bottomrule
\end{tabular}
\end{table}
There are more than 100 times more yellow cab pick-ups that happen in
Manhattan everyday than in Brooklyn. By comparing the average tip
percent in Table 3.2 and Table 3.3, we can observe that percent tips
paid in taxi zones with low pick-up numbers seem to be higher than
percent tips paid in taxi zones with high pick-up numbers.

\section{What features of taxi trips increase the percent tip amount
that passengers
pay?}\label{what-features-of-taxi-trips-increase-the-percent-tip-amount-that-passengers-pay}

So far, we have learned what pick-up zones offer the highest percent
tip. Now, we want to dig into the relationships between percent tip and
taxi-zone-specific variables.

\subsection{Does trip distance increase the percent tips paid by
passengers?}\label{does-trip-distance-increase-the-percent-tips-paid-by-passengers}

Will longer trips result in higher tip percent. It takes taxi drivers
more time to complete longer trips, so passengers might want to
compensate taxi drivers more. I personally pay higher percent of tips
for longer rides, so I believe trip distance has an impact on percentage
of tips paid.
\begin{verbatim}
                Estimate   Std. Error   t value Pr(>|t|)
(Intercept)  0.211325482 3.733048e-04 566.09360        0
avg_dis     -0.001052903 2.105014e-05 -50.01881        0
\end{verbatim}
Acoording to the simple linear regression result, trip distance does
have a negative significant impact on the percent of tips paid,
controlling for both pick-up and drop-off locations. This could be
caused by a psychological reason. Long trips cost more than short trips.
For a constant tip percent, the nominal value of tip amount cost more
for longer trips. For example, for a \$100 trip, 20\% tip costs \$20;
for a \$50 trip, 20\% tip costs \$10. Even though consumers are paying
the same percent amount of tips, \$20 is more expensive than \$10.
Therefore, consumers might decide to pay less percent tip for longer
trips.

\subsection{Do passengers pay more tips during rush
hours?}\label{do-passengers-pay-more-tips-during-rush-hours}

New York City Taxi Fare \& Limousine Commission has information on how
New York City taxi fare amount is calculated on their
\href{http://www.nyc.gov/html/tlc/html/passenger/taxicab_rate.shtml}{official
website}.

\subsubsection{Metered Fare Information}\label{metered-fare-information}
\begin{itemize}
\tightlist
\item
  Onscreen rate is `Rate \#01 -- Standard City Rate.'
\item
  The initial charge is \$2.50.
\item
  Plus 50 cents per 1/5 mile or 50 cents per 60 seconds in slow traffic
  or when the vehicle is stopped.
\item
  In moving traffic on Manhattan streets, the meter should ``click''
  approximately every four downtown blocks, or one block going
  cross-town (East-West).
\item
  There is a 50-cent MTA State Surcharge for all trips that end in New
  York City or Nassau, Suffolk, Westchester, Rockland, Dutchess, Orange
  or Putnam Counties.
\item
  There is a 30-cent Improvement Surcharge.
\item
  There is a daily 50-cent surcharge from 8pm to 6am.
\item
  There is a \$1 surcharge from 4pm to 8pm on weekdays, excluding
  holidays.
\item
  Passengers must pay all bridge and tunnel tolls.
\item
  Your receipt will show your total fare including tolls. Please take
  your receipt.
\item
  The driver is not required to accept bills over \$20.
\item
  Please tip your driver for safety and good service.
\item
  There are no charges for extra passengers or bags.
\end{itemize}
The metered fare rate information is collected from TLC rate of fare
webpage (N. T. staff, n.d.).

In taxi fare calculation, the only unknown variable is slow-traffic
time, and all other variables were collected by the meters installed on
each medallion taxi for each trip. It is reasonable to assume that for
trips with the same pick-up and drop-off locations, the longer the total
slow traffic time is, the longer the trip would take. Taxi drivers are
compensated for both the normal-speed trip distance and the time spent
in slow-traffic. According to the fare calculation algorithm, in moving
traffic on Manhattan streets, the meter should ``click'' approximately
every four downtown blocks, or one block going cross-town (East-West);
in slow traffic, the meter should ``click'' every 60 seconds. Therefore,
slow traffic increases the minute per mile ratio.

New York City has the worst traffic jams, and it has overtaken Miami to
be voted the U.S. city with the angriest and most aggressive drivers in
2009, according to a survey on road rage released on Tuesday. Bad
traffic also causes slow-traffic, and taxi drivers tend to get stuck in
traffic during rush hours (Reaney, 2009). Does minute per mile ratio
have an impact on the percent tip that passengers pay? Do passengers
compensate taxi drivers more during rush hours? Are passengers
sympathetic to taxi drivers for the time they spend in slow traffic?
\begin{verbatim}
                Estimate   Std. Error   t value Pr(>|t|)
(Intercept)  0.193453599 1.870711e-04 1034.1180        0
min_per_mile 0.002841351 3.134322e-05   90.6528        0
\end{verbatim}
As shown in the regression result, \texttt{min\_per\_mile} ratio does
have an positive impact on percent tips. Since trips with slow traffic
can be depicted by high minute per mile ratio, passengers do pay more
tips during rush hours.

Our analysis has proved that taxi passengers are sympathetic with the
drivers who have to suffer the cogestion in New York City, and taxi
drivers do get compensated more during rush hours.

\chapter{New York City Taxi Passengers}\label{chapter4}

\section{How long does it take passengers to get to JFK, La Guardia, and
Newark Airports from anywhere in New York City? When is the best time to
travel in order to avoid the
traffic?}\label{how-long-does-it-take-passengers-to-get-to-jfk-la-guardia-and-newark-airports-from-anywhere-in-new-york-city-when-is-the-best-time-to-travel-in-order-to-avoid-the-traffic}

We want to calculate the average number of minutes it takes to go to all
three airport from a specific taxi zone at every hour. First, we want to
focus on trips going to any of the three airports, JFK, LaGuardia, or
Newark Airport. We need to load trip records with destination as one of
the three airports from the MySQL connection we built.
\begin{Shaded}
\begin{Highlighting}[]
\NormalTok{to_jfk_trip <-}\StringTok{ }\NormalTok{taxi }\OperatorTok
\StringTok{  }\KeywordTok{tbl}\NormalTok{(}\StringTok{"yellow"}\NormalTok{) }\OperatorTok
\StringTok{  }\KeywordTok{filter}\NormalTok{(DOLocationID }\OperatorTok{==}\StringTok{ }\DecValTok{132}\NormalTok{) }\OperatorTok
\StringTok{  }\KeywordTok{collect}\NormalTok{(}\DataTypeTok{n =} \OtherTok{Inf}\NormalTok{)}

\NormalTok{to_lg_trip <-}\StringTok{ }\NormalTok{taxi }\OperatorTok
\StringTok{  }\KeywordTok{tbl}\NormalTok{(}\StringTok{"yellow"}\NormalTok{) }\OperatorTok
\StringTok{  }\KeywordTok{filter}\NormalTok{(DOLocationID }\OperatorTok{==}\StringTok{ }\DecValTok{138}\NormalTok{) }\OperatorTok
\StringTok{  }\KeywordTok{collect}\NormalTok{(}\DataTypeTok{n =} \OtherTok{Inf}\NormalTok{)}

\NormalTok{to_newark_trip <-}\StringTok{ }\NormalTok{taxi }\OperatorTok
\StringTok{  }\KeywordTok{tbl}\NormalTok{(}\StringTok{"yellow"}\NormalTok{) }\OperatorTok
\StringTok{  }\KeywordTok{filter}\NormalTok{(DOLocationID }\OperatorTok{==}\StringTok{ }\DecValTok{1}\NormalTok{) }\OperatorTok
\StringTok{  }\KeywordTok{collect}\NormalTok{(}\DataTypeTok{n =} \OtherTok{Inf}\NormalTok{)}
\end{Highlighting}
\end{Shaded}
Now we want to calculate the average amount of time it take from each
zone to one of the three airports during each hour.

So far, we have created three tables summarsing the average number of
minutes it takes to go to all three airports for every hour from
different taxi zones. It would be easier if we combine all three tables
and put information related to trip duration to all three airports in
the same table.
\begin{table}

\caption{\label{tab:unnamed-chunk-58}Average number of minutes it takes from Alphabet City, Manhattan to JFK Airport during different hours}
\centering
\begin{tabular}[t]{lcccc}
\toprule
  & PULocationID & hour & avg\_min & airport\\
\midrule
10 & 4 & 0 & 45.37000 & JFK\\
11 & 4 & 1 & 36.77500 & JFK\\
12 & 4 & 2 & 28.66000 & JFK\\
13 & 4 & 3 & 27.83350 & JFK\\
14 & 4 & 4 & 27.19490 & JFK\\
\addlinespace
15 & 4 & 5 & 28.68889 & JFK\\
16 & 4 & 6 & 34.25271 & JFK\\
17 & 4 & 7 & 38.13817 & JFK\\
18 & 4 & 8 & 41.59687 & JFK\\
19 & 4 & 9 & 35.39226 & JFK\\
20 & 4 & 10 & 36.22867 & JFK\\
\bottomrule
\end{tabular}
\end{table}
Table @ref(tab:three\_air) displays the average number of minutes it
takes from Alphabet City, Manhattan to JFK Airport during different
hours.

\subsection{Case Study: From Alphabet City, Manhattan to all three
airport}\label{case-study-from-alphabet-city-manhattan-to-all-three-airport}

Alphabet City, Manhattan has pick-up zone ID number 4. Let's take a look
at how much time is needed to travel to all three airports from taxi
zone No.4.
\begin{Shaded}
\begin{Highlighting}[]
\NormalTok{alphabet <-}\StringTok{ }\NormalTok{three_air}\OperatorTok
\StringTok{  }\KeywordTok{filter}\NormalTok{(PULocationID }\OperatorTok{==}\DecValTok{4}\NormalTok{)}
\end{Highlighting}
\end{Shaded}
\begin{figure}
\centering
\includegraphics{thesis_files/figure-latex/airport-vis-1.pdf}
\caption{\label{fig:airport-vis}Average number of minutes it takes from
Alphabet City, Manhattan to all three airports during different hours}
\end{figure}
According to the red line Figure \ref{fig:airport-vis}, it takes the
least time, less than 30 minutes, to travel from Alphabet City,
Manhattan to JFK Airport around 4 AM in the morning, and it takes the
most time, about 70 minutes, around 4 PM in the afternoon.

According to the green line, it takes the least time, about 20 minutes,
to travel to La Guardia Airport around 5 AM in the morning, and it takes
the most time, more than 40 minutes, around 5 PM in the afternoon.

As shown by the blue line, it takes the least time, a little less than
30 minutes, to travel to Newark Airport at 2 AM at midnight, and it
takes the most time, a little less than 80 minutes, around 6 PM in the
evening.

Being able to know the average time it takes to go to one of the
airports ahead, passengers can buy their flight tickets accordingly.

\subsection{A Shiny App: allowing users to choose a pick up zone of
their interest, and output the best time to travel from that zone to all
three airports in New
York}\label{a-shiny-app-allowing-users-to-choose-a-pick-up-zone-of-their-interest-and-output-the-best-time-to-travel-from-that-zone-to-all-three-airports-in-new-york}
\begin{center}\includegraphics[width=6.74in]{figure/shinyapp} \end{center}

This Shiny App helps passengers to estimate the amount of time that is
needed for them to travel to any one of the these three airports from
any New York City taxi zones.

\section{How does weather affect the number of taxi and Lyft
trips?}\label{how-does-weather-affect-the-number-of-taxi-and-lyft-trips}

On a snowy or rainy day, it is hard for passengers to find a yellow cab
on the street. Taxi drivers get paid at the same rate no matter how bad
the weather gets, so they tend to stay at home instead of going out to
work when the weather is bad. Uber drivers, however, get paid more on a
snowy or rainy day, since Uber uses a pricing model that takes the
number of Uber vehicles available on the street into account. When
weather is bad, fewer Uber vehicles are available on the street, so Uber
fare rate increases. Uber's pricing model gives Uber drivers an
incentive to keep working on ugly days. Lyft has a similar pricing model
to the one that Uber uses.

In this section, we want to study the number of pickups of yellow cab,
Uber, and Lyft. We compare number of pick-ups in each taxi zone in the
weeks of bad weather with previous weeks' total number of pick-ups to
see whether Uber drivers have an incentive to drive around the city more
when weather gets bad.

\textbf{Uber Weekly Data}
\begin{table}

\caption{\label{tab:unnamed-chunk-63}Uber 2017 Weekly Total Dispatched Trips}
\centering
\begin{tabular}[t]{ccc}
\toprule
Pickup Start Date & Pickup End Date & Uber Total Dispatched Trips\\
\midrule
2017-01-01 & 01/07/2017 & 2866569\\
2017-01-08 & 01/14/2017 & 3114792\\
2017-01-15 & 01/21/2017 & 3089595\\
2017-01-22 & 01/28/2017 & 3299763\\
2017-01-29 & 02/04/2017 & 3224451\\
\addlinespace
2017-02-05 & 02/11/2017 & 3310481\\
2017-02-12 & 02/18/2017 & 3456042\\
2017-02-19 & 02/25/2017 & 3194805\\
2017-02-26 & 03/04/2017 & 3533347\\
2017-03-05 & 03/11/2017 & 3614559\\
\bottomrule
\end{tabular}
\end{table}
\subsubsection{Yellow Cab Weekly Data}\label{yellow-cab-weekly-data}
\begin{table}

\caption{\label{tab:unnamed-chunk-66}Yellow Taxi 2017 Weekly Total Dispatched Trips}
\centering
\begin{tabular}[t]{ccc}
\toprule
Pickup Start Date & Pickup End Date & Yellow Total Dispatched Trips\\
\midrule
2017-01-01 & 01/07/2017 & 2044643\\
2017-01-08 & 01/14/2017 & 2230950\\
2017-01-15 & 01/21/2017 & 2219214\\
2017-01-22 & 01/28/2017 & 2307122\\
2017-01-29 & 02/04/2017 & 2331749\\
\addlinespace
2017-02-05 & 02/11/2017 & 2181622\\
2017-02-12 & 02/18/2017 & 2387399\\
2017-02-19 & 02/25/2017 & 2225850\\
2017-02-26 & 03/04/2017 & 2464800\\
2017-03-05 & 03/11/2017 & 2456285\\
\bottomrule
\end{tabular}
\end{table}
In this section, we use New York City Yellow Taxi and Uber data to
calculate the number of trips occurred in each week.

\subsection{Case Study: March 14th, 2017 Snow
Storm}\label{case-study-march-14th-2017-snow-storm}

There are two commonly known bad weather consitions, rainy and snowy
days. Let's first focus on snowstorm. On March 14th, 2017, a snow storm
brought seven inches of snow to New York City. \textbf{Yellow Taxi}
\begin{verbatim}
  Pickup Start Date Yellow Total Dispatched Trips Pickup End Date
1        2017-03-05                       2456285      03/11/2017
2        2017-03-12                       2066285      03/18/2017
\end{verbatim}
\begin{Shaded}
\begin{Highlighting}[]
\NormalTok{(}\DecValTok{2066285}\OperatorTok{-}\DecValTok{2456285}\NormalTok{)}\OperatorTok{/}\DecValTok{2456285}
\end{Highlighting}
\end{Shaded}
\begin{verbatim}
[1] -0.1587764
\end{verbatim}
\textbf{Uber}
\begin{verbatim}
# A tibble: 0 x 3
# Groups:   Pickup Start Date [0]
# ... with 3 variables: Pickup Start Date <chr>, Pickup End Date <chr>,
#   Uber Total Dispatched Trips <int>
\end{verbatim}
\begin{Shaded}
\begin{Highlighting}[]
\NormalTok{(}\DecValTok{3430189}\OperatorTok{-}\DecValTok{3614559}\NormalTok{)}\OperatorTok{/}\DecValTok{3614559}
\end{Highlighting}
\end{Shaded}
\begin{verbatim}
[1] -0.05100761
\end{verbatim}
In this case, we observe that the percent decline in Uber's total number
of pick-ups is 10\% less than the percent decline in Yellow Taxi's total
number of dropp-off. Even though the total number of Uber pick-ups did
not increase, Uber's pricing model definitely was able to keep more
drivers in the market on a snowy day.

\subsection{Case Study: Impact of Precipitation on Taxi
Rides}\label{case-study-impact-of-precipitation-on-taxi-rides}

People living in New York might have noticed that it is hard to find a
taxi on the street when it rains. Economists have studied this phenomena
for a long time, and an analysis studied the correlationship between
taxi movement and hourly rainfall data in Central Park from 2009 to 2013
has found that there is no significant correlation between a driver's
hourly wage and rain in the city, which implies that drivers don't earn
more when it's raining (Jaffe, 2014).

I got access to the 2017 daily Central Park weather data from the
National Climatic Data Center by submitting a Climate Data Online
request (Appendix D) to National Centers for Environmental Information
(Environmental Information staff, n.d.), and joined it to the 2017 taxi
data to study relationship between rainfall and taxi rides.

First, we generate a list of total amount of daily rainfall in New York
City and we pick the 10 weeks that have the most rainfall in 2017.
\begin{table}

\caption{\label{tab:unnamed-chunk-72}10 weeks that have the most rainfall in 2017}
\centering
\begin{tabular}[t]{ccc}
\toprule
Pickup Start Date & Pickup End Date & Weekly Rainfall\\
\midrule
2017-06-18 & 2017-06-25 & 7.00\\
2017-04-30 & 2017-05-07 & 6.71\\
2017-10-29 & 2017-11-05 & 6.16\\
2017-07-02 & 2017-07-09 & 5.56\\
2017-03-26 & 2017-04-02 & 5.11\\
\addlinespace
2017-01-22 & 2017-01-29 & 3.99\\
2017-08-13 & 2017-08-20 & 3.95\\
2017-06-11 & 2017-06-18 & 3.91\\
2017-04-02 & 2017-04-09 & 3.17\\
2017-04-23 & 2017-04-30 & 2.88\\
\bottomrule
\end{tabular}
\end{table}
We then find the weekly total number of dispatched yellow taxi trips of
the 10 weeks with the srtat date listed above.
\begin{table}

\caption{\label{tab:unnamed-chunk-74}10 weeks that have the most rainfall in 2017 and the total number of dispatched yellow taxi trips in those weeks}
\centering
\begin{tabular}[t]{ccccc}
\toprule
Pickup Date & Dispatched Trips & Last Week Date & Last Week Trips & \%Change Trips\\
\midrule
2017-06-18 & 2231205 & 2017-06-11 & 2285958 & -2.3951884\\
2017-04-30 & 2386559 & 2017-04-23 & 2394329 & -0.3245168\\
2017-10-29 & 2266196 & 2017-10-22 & 2267693 & -0.0660142\\
2017-07-02 & 1664159 & 2017-06-25 & 2038406 & -18.3597870\\
2017-03-26 & 2341096 & 2017-03-19 & 2272369 & 3.0244648\\
\addlinespace
2017-01-22 & 2307122 & 2017-01-15 & 2219214 & 3.9612223\\
2017-08-13 & 1871668 & 2017-08-06 & 1929860 & -3.0153483\\
2017-06-11 & 2285958 & 2017-06-04 & 2313236 & -1.1792139\\
2017-04-02 & 2414700 & 2017-03-26 & 2341096 & 3.1439975\\
2017-04-23 & 2394329 & 2017-04-16 & 2337161 & 2.4460446\\
\bottomrule
\end{tabular}
\end{table}
We also need to add the weekly total number of dispatched Uber trips of
the 10 weeks with the most rainfall.
\begin{table}

\caption{\label{tab:unnamed-chunk-76}10 weeks that have the most rainfall in 2017 and the total number of dispatched Uber trips in those weeks}
\centering
\begin{tabular}[t]{ccccc}
\toprule
Pickup Date & Dispatched Trips & Last Week Date & Last Week Trips & \%Change Trips\\
\midrule
2017-06-18 & 3654932 & 2017-06-11 & 3658220 & -0.0898798\\
2017-04-30 & 3546893 & 2017-04-23 & 3606408 & -1.6502570\\
2017-10-29 & 4317572 & 2017-10-22 & 4193611 & 2.9559489\\
2017-07-02 & 3212582 & 2017-06-25 & 3406814 & -5.7012798\\
2017-03-26 & 3541624 & 2017-03-19 & 3425475 & 3.3907414\\
\addlinespace
2017-01-22 & 3299763 & 2017-01-15 & 3089595 & 6.8024450\\
2017-08-13 & 3599772 & 2017-08-06 & 3584023 & 0.4394224\\
2017-06-11 & 3658220 & 2017-06-04 & 3622252 & 0.9929734\\
2017-04-02 & 3443444 & 2017-03-26 & 3541624 & -2.7721746\\
2017-04-23 & 3606408 & 2017-04-16 & 3427564 & 5.2178165\\
\bottomrule
\end{tabular}
\end{table}
We combine the percentage change in total number of dispatched trips of
yellow taxi and Uber, and we compare the result.
\begin{table}

\caption{\label{tab:unnamed-chunk-78}The percentage change in total number of dispatched trips comparing to the previous weeks of yellow taxi and Uber}
\centering
\begin{tabular}[t]{ccccc}
\toprule
Pickup Start Date & Weekly Rainfall & Last Week Date & uber & yellow\\
\midrule
2017-06-18 & 7.00 & 2017-06-11 & -0.0898798 & -2.3951884\\
2017-04-30 & 6.71 & 2017-04-23 & -1.6502570 & -0.3245168\\
2017-10-29 & 6.16 & 2017-10-22 & 2.9559489 & -0.0660142\\
2017-07-02 & 5.56 & 2017-06-25 & -5.7012798 & -18.3597870\\
2017-03-26 & 5.11 & 2017-03-19 & 3.3907414 & 3.0244648\\
\addlinespace
2017-01-22 & 3.99 & 2017-01-15 & 6.8024450 & 3.9612223\\
2017-08-13 & 3.95 & 2017-08-06 & 0.4394224 & -3.0153483\\
2017-06-11 & 3.91 & 2017-06-04 & 0.9929734 & -1.1792139\\
2017-04-02 & 3.17 & 2017-03-26 & -2.7721746 & 3.1439975\\
2017-04-23 & 2.88 & 2017-04-16 & 5.2178165 & 2.4460446\\
\bottomrule
\end{tabular}
\end{table}
Besides the week of April 30th, 2017, all other weeks have higher
increases in the number of total dispatched trips of Uber or lower
declines in the number of weekly Uber trips. Therefore, on rainy days,
Uber drivers tend to increase the number of trips they drive at a higher
rate.

Uber passengers have to pay higher rate of fare on rainy days becasue of
Uber's pricing model. Since taxi drivers do not get paid more on rainy
days, they tend to work less than Uber drivers, which limits the options
for passengers. Passengers sometimes have to choose the more costly Uber
instead. New York City TLC could modify the rate of fare on rainy or
snowy days to incentive taxicab drivers to pick up more trips in order
to make taking a street hail vechcle on average more affordable on rainy
days for passengers.

\chapter{New York City Taxi Fare \& Limousine
Commission}\label{chapter5}

\section{Should there be a flat rate between Manhattan and John F.
Kennedy International
Airport?}\label{should-there-be-a-flat-rate-between-manhattan-and-john-f.-kennedy-international-airport}

Why is there a flat rate to and from JFK airport and any location in
Manhattan? Why is the flat rate \$52? Does TLC make profit from the \$52
flat rate? Does \$52 reduce the cogestion on the road to JFK airport and
make taking a train a more preferable choice? The New York City taxi
trip records can reveal the answers to these questions.

Imagine it's your first time travelling to New York City, and you
decided to live in a hotel in Manhattan. Since you might not know much
about the city, the \$52 flat rate is nice for you, and it incentivizes
you to take taxi to the JFK Airport. If there is no flat rate, there is
uncertainty in how much someone needs to pay to take a taxi to JFK, and
tourists might instead choose to take the train, even though taking a
train would cost them more time and inconvenience.

Additionally, people living in most parts of Manhattan would have paid
more than \$52 to take a taxi to go to the JFK Airport. The higher the
taxi fare is, the less the demand for taxi will be. Therefore, having a
flat rate might help taxi drivers to get more trips from Manhattan to
JFK Airport.

\section{Passengers departing from Manhattan benefit from the \$52 flat
rate}\label{passengers-departing-from-manhattan-benefit-from-the-52-flat-rate}

If there is no flat rate between JFK and Manhattan, how much would
passengers pay for the distance they travelled between JFK Airport AND
Manhattan? And how much more or less should they have paid comparing to
the \$52 flat rate?

In this study, we are only interested in yellow taxi trip between
Manhattan and JFK Airport.
\begin{Shaded}
\begin{Highlighting}[]
\NormalTok{to_jkf <-}\StringTok{ }\NormalTok{taxi }\OperatorTok
\StringTok{  }\KeywordTok{tbl}\NormalTok{(}\StringTok{"yellow"}\NormalTok{) }\OperatorTok
\StringTok{  }\KeywordTok{filter}\NormalTok{(DOLocationID }\OperatorTok{==}\StringTok{ }\DecValTok{132}\NormalTok{) }\OperatorTok
\StringTok{  }\KeywordTok{collect}\NormalTok{(}\DataTypeTok{n =} \OtherTok{Inf}\NormalTok{)}

\NormalTok{from_jfk <-}\StringTok{ }\NormalTok{taxi }\OperatorTok
\StringTok{  }\KeywordTok{tbl}\NormalTok{(}\StringTok{"yellow"}\NormalTok{) }\OperatorTok
\StringTok{  }\KeywordTok{filter}\NormalTok{(PULocationID }\OperatorTok{==}\StringTok{ }\DecValTok{132}\NormalTok{) }\OperatorTok
\StringTok{  }\KeywordTok{collect}\NormalTok{(}\DataTypeTok{n =} \OtherTok{Inf}\NormalTok{)}
\end{Highlighting}
\end{Shaded}
\begin{verbatim}

Read 72.7% of 577783 rows
Read 577783 rows and 20 (of 20) columns from 0.064 GB file in 00:00:03
\end{verbatim}
\begin{verbatim}

Read 4.0% of 1491842 rows
Read 32.2% of 1491842 rows
Read 53.0% of 1491842 rows
Read 55.0% of 1491842 rows
Read 83.8% of 1491842 rows
Read 85.1% of 1491842 rows
Read 1491842 rows and 20 (of 20) columns from 0.164 GB file in 00:00:10
\end{verbatim}
\subsection{Trips from Manhattan to JFK
Airport}\label{trips-from-manhattan-to-jfk-airport}

We first focus on all the trips that departed in Manhattan and went to
JFK Airport, and then we calculate the estmated fare amount that the
passengers should have paid based on the distance travelled from each
pick-up point to JFK Airport based on the fare rate suggested by TLC for
each pick-up zone.

Here is a map of estmated fare amount calculated by taking the average
of all estimated fare amounts from the same pick-up zone to JFK Airport
based on the fare rate suggested by TLC for each pick-up zone.
\begin{figure}

{\centering \includegraphics[width=4.96in]{figure/to_jkf_fare_vis} 

}

\caption{Estmated fare amount from the each pick-up zone to JFK Airport}\label{fig:to-jkf-fare-vis}
\end{figure}
According to the map, trips from Midtown on average cost less than trips
from other taxi zones in Manhattan.

\subsection{Which taxi zones would pay more than \$52 without the flat
rate?}\label{which-taxi-zones-would-pay-more-than-52-without-the-flat-rate}
\begin{table}

\caption{\label{tab:unnamed-chunk-89}Ten pick-up zones with the highest avergae fare from Manhattan to JKF Airport}
\centering
\begin{tabular}[t]{cccc}
\toprule
avg\_est\_fare & avg\_est\_diff & Borough & Zone\\
\midrule
64.03150 & 11.844558 & Manhattan & Battery Park City\\
63.98256 & 9.970366 & Manhattan & Inwood\\
62.97567 & 10.892992 & Manhattan & Washington Heights North\\
61.99327 & 9.889636 & Manhattan & Battery Park\\
60.49388 & 8.278941 & Manhattan & Washington Heights South\\
\addlinespace
60.18006 & 8.107309 & Manhattan & Upper West Side South\\
59.74384 & 7.511991 & Manhattan & World Trade Center\\
59.31411 & 7.058534 & Manhattan & Meatpacking/West Village West\\
59.24692 & 7.200516 & Manhattan & Lincoln Square West\\
59.13439 & 7.083517 & Manhattan & Upper West Side North\\
\bottomrule
\end{tabular}
\end{table}
We computed the average fare paid by passengers for trips going from
each taxi zone in Manhattan to JFK Airport in Table
@ref(tab:to\_jkf\_zone\_above).

Let's visualize the taxi zones that would have costed more than the \$52
flat rate.
\begin{figure}

{\centering \includegraphics[width=4.96in]{figure/to_jfk_fare_above_vis} 

}

\caption{Pick-up Zones that cost more than the 52 US Dollar flat rate}\label{fig:to-jkf-fare-above-vis}
\end{figure}
Therefore, passengers from places in Manhattan besides Midtown, East
Village, and some parts of Lower Manhattan benefit from the \$52 flat
rate. However, people living in Midtown, East Village, and some parts of
Lower Manhattan might be relatively more indifferent to the price of
taxi. Instead, they probably put more emphasis on convenience and time.
\begin{verbatim}
[1] 2.825257
\end{verbatim}
On average people travel from Manhattan pay \texttt{meanfare} less with
the \$52 flat rate policy. Therefore, passengers overall benefit from
the \$52 flat rate policy.

\section{Are taxi drivers happy when a passenger wants to go to JFK
Airport from
Manhattan?}\label{are-taxi-drivers-happy-when-a-passenger-wants-to-go-to-jfk-airport-from-manhattan}

Everytime I travel to New York City, I always take Yellow cabs to go
around the city. It seemed to me that the cab drivers were always happy
when they heard me telling them that I needed to go to the JFK Airport
from Manhattan. Are taxi drivers happy when their passengers want to go
to JFK Airport from Manhattan? In this section, we study the hourly wage
of taxi drivers for different trips they completed, and we investigate
whether taxi driver hourly wage from Manhattan to JFK Airport is higher
than other trips.
\begin{verbatim}
[1] 63.29116
\end{verbatim}
The average hourly wage of taxi drivers calculated by using all trips
excluding the ones going from Manhattan to JFK Airport is
\texttt{mean\_overall}.
\begin{verbatim}
[1] 69.05252
\end{verbatim}
The average hourly wage of taxi drivers calculated by using trips going
from Manhattan to JFK Airport is \texttt{mean\_jfk}. \texttt{mean\_jfk}
dollar per hour is higher than \texttt{mean\_overall} dollar per hour,
which means that on average taxi drivers driving from Manhattan to JFK
Airport have an hourly wage that is about \$6 higher than the hourly
wage of taxi drivers doing other trips.

\subsection{How much on average would taxi driver make on their way back
from JFK
Airport?}\label{how-much-on-average-would-taxi-driver-make-on-their-way-back-from-jfk-airport}

Since a taxi driver waiting in line to pickup passengers at JFK Airport
could be directed back to anywhere in the city. We calculate the average
taxi fare amount that a taxi driver would get paid for a trip from JFK
Airport to any part of the city.

\subsubsection{What are the most popular drop-off locations for
passengers departing from JFK
Airport?}\label{what-are-the-most-popular-drop-off-locations-for-passengers-departing-from-jfk-airport}
\begin{table}

\caption{\label{tab:unnamed-chunk-96}5 most popular destinations in Manhattan}
\centering
\begin{tabular}[t]{ccccc}
\toprule
Borough & Zone & num\_trips & avg\_fare & avg\_duration\\
\midrule
Manhattan & Times Sq/Theatre District & 59419 & 69.80599 & 55.92389\\
Manhattan & Midtown East & 40513 & 69.40195 & 47.42096\\
Manhattan & Murray Hill & 40071 & 69.91174 & 43.66998\\
Manhattan & Midtown South & 38890 & 70.11065 & 48.34342\\
Manhattan & Midtown Center & 36405 & 69.64272 & 52.62410\\
\bottomrule
\end{tabular}
\end{table}
Table @ref(tab:from\_jkf\_zone) shows that Times Square is the most
popular destination for passengers coming from the JFK Airport in 2017!
\begin{figure}

{\centering \includegraphics[width=4.96in]{figure/from_jkf_num_trips} 

}

\caption{Zones that cost more than the 52 US Dollar flat rate}\label{fig:from-jkf-num-trips}
\end{figure}
According to the Figure @\ref(fig:from-jkf-num-trips), Manhattan is the
most popular destination for passengers departing from JFK Airport.
\begin{verbatim}
# A tibble: 2 x 2
  Manhattan all_trips
      <dbl>     <int>
1         0    521476
2         1    970366
\end{verbatim}
According to the summary, the total amount of trips from JFK Airport to
Manhattan is about \texttt{total\_trip}\% of the total number of trips
travelling from JFK Airport to all other Borough. Therefore, it is very
likely for taxi drivers to get passengers who want to go to Manhattan
with a flat rate of \$52. In this case, a round trip to and from JFK
Airport is worthwhile Therefore, taxi drivers should be pretty happy
when their passengers are going to JFK Airport from Manhattan.

\subsubsection{What's the average fare to each dropp-off zone from JFK
Airport?}\label{whats-the-average-fare-to-each-dropp-off-zone-from-jfk-airport}

We can use a map to visualize the distribution of average fare amount
needed to travel from JFK Airport to any taxi zone in New York City.
\begin{table}

\caption{\label{tab:unnamed-chunk-100}10 most popular taxi drop-off zones from JFK Airport with the corresponding average fare amount}
\centering
\begin{tabular}[t]{ccccc}
\toprule
Borough & Zone & num\_trips & avg\_fare & avg\_duration\\
\midrule
Manhattan & Times Sq/Theatre District & 59419 & 69.80599 & 55.92389\\
Manhattan & Midtown East & 40513 & 69.40195 & 47.42096\\
Manhattan & Murray Hill & 40071 & 69.91174 & 43.66998\\
Manhattan & Midtown South & 38890 & 70.11065 & 48.34342\\
Manhattan & Midtown Center & 36405 & 69.64272 & 52.62410\\
\addlinespace
Manhattan & Clinton East & 35297 & 69.20170 & 55.78806\\
Manhattan & Midtown North & 34538 & 68.30039 & 55.80455\\
Brooklyn & Park Slope & 27219 & 60.96428 & 45.75234\\
Manhattan & East Village & 26595 & 66.98102 & 45.45684\\
Manhattan & Upper West Side South & 24723 & 69.94912 & 50.87777\\
\bottomrule
\end{tabular}
\end{table}
\begin{figure}

{\centering \includegraphics[width=4.96in]{figure/from_jkf_fare_vis} 

}

\caption{Zones that cost more than the 52 US Dollar flat rate}\label{fig:from-jkf-fare-vis}
\end{figure}
As we expected, the red shades are smoothly distributed, since taxi
zones that are futher away should cost more to get there.

\subsubsection{How much on average would taxi driver make on their way
back from JFK
Airport?}\label{how-much-on-average-would-taxi-driver-make-on-their-way-back-from-jfk-airport-1}
\begin{verbatim}
[1] 62.69673
\end{verbatim}
On average, taxi drivers would get paid for \texttt{mean\_fare} for a
trip from the JFK Airport to any taxi zone in New York City.
\begin{verbatim}
[1] 63.14558
\end{verbatim}
The average hourly wage of taxi drivers calculated by using all trips
excluding the ones departing from JFK Airport is \texttt{mean\_no\_jfk}.
\begin{verbatim}
[1] 94.65196
\end{verbatim}
The average hourly wage of taxi drivers calculated by using trips
departing from JFK Airport is \texttt{mean\_from\_jfk\_wage} dollar per
hour. \texttt{mean\_from\_jfk\_wage} dollar per hour is higher than
\texttt{mean\_no\_jfk} dollar per hour, which means that on average taxi
drivers going from JFK Airport to any taxi zones in New York City have
an hourly wage that is more than \$30 higher than the hourly wage of
taxi drivers doing other trips.

In conclusion, taxi drivers' hourly wages are higher when they drive
from Manhattan to JFK Airport. On their way back, they make on average
\texttt{mean\_from\_jfk\_wage} dollar per hour, which is much higher
than the average hourly wage of yellow cab drivers. Therefore, taxicab
drivers feel happy when they pick up passengers who want to go to JFK
Airport in Manhattan.

\chapter{Conclusion}\label{chapter5}

In this Honors thesis, we present a more efficient and easy-to-use way
for users to retrieve trip records information of both New York City
taxi and other ride-sharing services, such as Uber and Lyft, in New York
City.

By analyzing trip records of New York City's yellow taxi, we found
answers to questions that are commonly asked by taxi drivers,
passengers, and TLC officials.

We found which taxi zones have passengers who offer the highest percent
of tips, and we proved that taxi drivers do get compensated more during
rush hours. We are able to help passengers to know the average time it
takes to go to one of the three airports in New York City so that
passengers can plan their trips accordingly. We also found that the \$52
flat rate between Manhattan and JFK Airport is beneficial for the
passengers, because it is cheaper than the average amount of fare that
passenger would need to pay without the flat rate.

\section{Future Research}\label{future-research}

For future study, we would love to investigate the sharp decline in the
consumption of NYC yellow cab after e-hail services were introduced into
the NYC ride-hail market. We also want to study what the impact of
introducing new GPS and entertainment system is on the number of rides.
The global product and marketing at Verifone, Jason Gross, said that,
``We like to say that we provide what Uber says it provides.'' With the
raised expectation among rides caused by Uber and Lyft, yellow taxi
industry need to respond quickly (Hawkins, 2016). How does the market
react to the newly installed entertainment system? Has the market share
of yellow cab rebounded since 2016? By looking into the patterns in
market shares, it might be possible for me to predict the future market
share distribution and find out what features of ride-hail
transportation are the ones that affect market share distribution the
most.

\appendix

\chapter{Utility Function}\label{utility-function}

This utility function was written to shortened the source code in ETL
\texttt{etl\_extract.etl\_nyctaxi()} function.
\begin{Shaded}
\begin{Highlighting}[]
\NormalTok{download_nyc_data <-}\StringTok{ }\ControlFlowTok{function}\NormalTok{(obj, url, years, n, names, ...) \{}
\NormalTok{  url <-}\StringTok{ }\KeywordTok{paste0}\NormalTok{(url, }\StringTok{"?years="}\NormalTok{, years, }\StringTok{"&$limit="}\NormalTok{, n)}
\NormalTok{  lcl <-}\StringTok{ }\KeywordTok{file.path}\NormalTok{(}\KeywordTok{attr}\NormalTok{(obj, }\StringTok{"raw"}\NormalTok{), names)}
\NormalTok{  downloader}\OperatorTok{::}\KeywordTok{download}\NormalTok{(url, }\DataTypeTok{destfile =}\NormalTok{ lcl, ...)}
\NormalTok{  lcl}
\NormalTok{\}}
\end{Highlighting}
\end{Shaded}
\chapter{Data Dictionary -- Yellow
Taxi}\label{data-dictionary-yellow-taxi}
\begin{figure}
  \centering
  \includegraphics[width=4in]{figure/data_dictionary_trip_records_yellow.pdf}
  \caption{Data Dictionary -- Yellow Taxi Trips Records}
\end{figure}
\chapter{Appendix C: FOIL Request}\label{appendix-c-foil-request}
\begin{figure}
  \centering
  \includegraphics[width=4in]{figure/appendix_foil_form_doc.pdf}
  \caption{FOIL Request}
\end{figure}
\chapter{NOAA Climate Data Request}\label{noaa-climate-data-request}
\begin{figure}
  \centering
  \includegraphics[width=4in]{figure/app_noaa_request.pdf}
  \caption{NOAA Climate Data Request}
\end{figure}
\begin{figure}
  \centering
  \includegraphics[width=4in]{figure/app_noaa_com.pdf}
  \caption{NOAA Climate Data Order Compeleted}
\end{figure}
\backmatter

\chapter*{References}\label{references}
\addcontentsline{toc}{chapter}{References}

\markboth{References}{References}

\noindent

\setlength{\parindent}{-0.20in} \setlength{\leftskip}{0.20in}
\setlength{\parskip}{8pt}

\hypertarget{refs}{}
\hypertarget{ref-pkgetl}{}
Baumer, B. S. (2017). A grammar for reproducible and painless
extract-transform-load operations on medium data. \emph{arXiv},
\emph{8}(23), 1--24. Retrieved from
\url{https://arxiv.org/abs/1708.07073}

\hypertarget{ref-baumer2014}{}
Baumer, B., Cetinkaya-Rundel, M., Bray, A., Loi, L., \& Horton, N. J.
(2014). R markdown: Integrating a reproducible analysis tool into
introductory statistics. \emph{TISE}.

\hypertarget{ref-furfaro2016}{}
Danielle Furfaro, S. C., \& Fears, D. (2016, December). NYC is already
tired of Christmas and Donald Trump. New York Post. Retrieved from
\url{https://nypost.com/2016/12/01/nyc-is-already-tired-of-christmas-and-donald-trump/}

\hypertarget{ref-pkgdatatable}{}
Dowle, M., \& Srinivasan, A. (2017). \emph{Data.table: Extension of
`data.frame`}. Retrieved from
\url{https://CRAN.R-project.org/package=data.table}

\hypertarget{ref-noaa}{}
Environmental Information staff, N. C. for. (n.d.). Climate Data Online.
National Centers for Environmental Information. Retrieved from
\url{https://www.ncdc.noaa.gov/cdo-web/}

\hypertarget{ref-ubernyc}{}
Griswold, A. (2015, November). Uber Won New York. Slate Business.
Retrieved from
\url{http://www.slate.com/articles/business/moneybox/2015/11/uber_won_new_york_city_it_only_took_five_years.html}

\hypertarget{ref-pkglubridate}{}
Grolemund, G., \& Wickham, H. (2011). Dates and times made easy with
lubridate. \emph{Journal of Statistical Software}, \emph{40}(3), 1--25.
Retrieved from \url{http://www.jstatsoft.org/v40/i03/}

\hypertarget{ref-guerrini2015}{}
Guerrini, F. (2015, April). Which Is Cheaper To Use In NYC: Uber Or A
Taxi? Big Data Will Solve The Dilemma. Forbes Tech. Retrieved from
\url{https://www.forbes.com/sites/federicoguerrini/2015/04/09/living-in-new-york-this-app-will-tell-you-which-is-cheaper-uber-or-a-taxi/2/\#26bc29904023}

\hypertarget{ref-Rebecca2015}{}
Harshbarger, R. (2015, June). City puts biased drivers who refuse rides
on notice with new video. New York Post. Retrieved from
\url{https://nypost.com/2015/06/09/city-puts-biased-taxi-drivers-on-notice/}

\hypertarget{ref-andrew2016}{}
Hawkins, A. J. (2016, September). Yellow taxis have a new weapon in
their war against uber: Gadgets. The Verge. Retrieved from
\url{https://www.theverge.com/2016/9/26/13035642/nyc-taxi-cab-android-touchscreen-tablet-verifone}

\hypertarget{ref-hu2017}{}
Hu, W. (2017, January). Yellow Cab, Long a Fixture of City Life, Is for
Many a Thing of the Past. The New York Times. Retrieved from
\url{https://www.nytimes.com/2017/01/15/nyregion/yellow-cab-long-a-fixture-of-city-life-is-for-many-a-thing-of-the-past.html}

\hypertarget{ref-citylab}{}
Jaffe, E. (2014, October). Why New Yorkers Can't Find a Taxi When It
Rains. CITYLAB. Retrieved from
\url{https://www.citylab.com/environment/2014/10/why-new-yorkers-cant-find-a-taxi-when-it-rains/381652/}

\hypertarget{ref-foirequest}{}
Li, W. (2018, February). FOIL request. NYC TLC.

\hypertarget{ref-pkgnyctaxi}{}
Li, W. P., Baumer, B., \& Trang Le. (2017). \emph{Nyctaxi: Accessing new
york city taxi data}. Retrieved from
\url{http://github.com/beanumber/nyctaxi}

\hypertarget{ref-reaney2009}{}
Reaney, P. (2009, June). New York Drivers Named Most Aggressive, Angry
in U.S. Reuters. Retrieved from
\url{https://www.reuters.com/article/us-driving-roadrage-life/new-york-drivers-named-most-aggressive-angry-in-u-s-idUSTRE55F1J720090616}

\hypertarget{ref-schneider2015}{}
Schneider, T. W. (2015, November). Analyzing 1.1 Billion NYC Taxi and
Uber Trips, with a Vengeance. Todd W. Schneider. Retrieved from
\url{http://toddwschneider.com/posts/analyzing-1-1-billion-nyc-taxi-and-uber-trips-with-a-vengeance/}

\hypertarget{ref-cran}{}
staff, C. (n.d.). The Comprehensive R Archive Network. Retrieved from
\url{https://cran.r-project.org/index.html}

\hypertarget{ref-uberx}{}
staff, H. (2018, January). Uber's 4 Basic Level of Service. HyreCar.
Retrieved from
\url{https://hyrecar.com/blog/difference-between-uber-cars/}

\hypertarget{ref-datalyft}{}
staff, N. O. (2015a). LYFT Data. NYC OpenData. Retrieved from
\url{https://data.cityofnewyork.us/Transportation/LYFT-Data/juxc-sutg/data}

\hypertarget{ref-datauberweek}{}
staff, N. O. (2015b). Uber Trips NYC 2016. NYC OpenData. Retrieved from
\url{https://data.cityofnewyork.us/Transportation/Uber-Trips-NYC-2016/gt3n-7ri6/data}

\hypertarget{ref-datayellowmonth}{}
staff, N. T. (2009a). TLC Aggregated Reports. NYC Taxi \& Limousine
Commission. Retrieved from
\url{http://www.nyc.gov/html/tlc/html/technology/aggregated_data.shtml}

\hypertarget{ref-datayellow}{}
staff, N. T. (2009b). TLC Trip Record Data. NYC Taxi \& Limousine
Commission. Retrieved from
\url{http://www.nyc.gov/html/tlc/html/about/trip_record_data.shtml}

\hypertarget{ref-datauber}{}
staff, N. T. (2009c). TLC Trip Record Data. NYC Taxi \& Limousine
Commission. Retrieved from
\url{http://www.nyc.gov/html/tlc/html/about/trip_record_data.shtml}

\hypertarget{ref-greentaxi}{}
staff, N. T. (2009d). Your guide to Boro Taxis. NYC Taxi \& Limousine
Commission. Retrieved from
\url{http://www.nyc.gov/html/tlc/html/passenger/shl_passenger.shtml}

\hypertarget{ref-tlcfarerate}{}
staff, N. T. (n.d.). NYC tlc Taxicab Rate of Fare. NYC TLC. Retrieved
from \url{http://www.nyc.gov/html/tlc/html/passenger/taxicab_rate.shtml}

\hypertarget{ref-appone}{}
staff, O. (2015). OpenStreetCab. University of Cambridge UK Computer
Laboratory, University of Namur Belgium, Complexity; Networks group.
Retrieved from \url{https://www.openstreetcab.com}

\hypertarget{ref-wikipediataxi}{}
staff, W. (2018a, April). Taxicabs of New York City. \emph{Wikipedia}.
Wikimedia Foundation. Retrieved from
\url{https://en.wikipedia.org/wiki/Taxicabs_of_New_York_City}

\hypertarget{ref-wikipedia}{}
staff, W. (2018b, March). Extract, transform, load. \emph{Wikipedia}.
Wikimedia Foundation. Retrieved from
\url{https://en.wikipedia.org/wiki/Extract,_transform,_load}

\hypertarget{ref-sugar2017}{}
Sugar, R. (2017, January). Uber and Lyft cars now outnumber yellow cabs
in NYC 4 to 1. Curbed New York. Retrieved from
\url{https://ny.curbed.com/2017/1/17/14296892/yellow-taxi-nyc-uber-lyft-via-numbers}

\hypertarget{ref-emma2017}{}
Whitford, E. (2017, October). Daily Uber Trips Have Officially
Outstripped Taxi Trips. Gothamist. Retrieved from
\url{http://gothamist.com/2017/10/13/uber_taxis_nyc.php}

\hypertarget{ref-pkgreadr}{}
Wickham, H., Hester, J., \& Francois, R. (2017). \emph{Readr: Read
rectangular text data}. Retrieved from
\url{https://CRAN.R-project.org/package=readr}

\hypertarget{ref-zhang2017}{}
Zhang, W. (2017, May). Improving access to open-source data about the
nyc bike sharing system (Citi Bike). Smith College.


% Index?

\end{document}
